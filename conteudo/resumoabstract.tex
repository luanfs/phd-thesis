\palavrachave{Núcleo dinâmico da atmosfera, esfera cubada, volumes finitos, dimension splitting, ponto de partida, corretor de massa, equação de advecção, equação de águas rasas}

\keyword{Dynamical core, cubed-sphere, finite-volume, dimension splitting, departure point, mass fixer, advection equation, shallow-water equations}

% O resumo é obrigatório, em português e inglês. Estes comandos também
% geram automaticamente a referência para o próprio documento, conforme
% as normas sugeridas da USP.
\resumo{
O núcleo dinâmico FV3 do GFDL-NOAA-EUA, originalmente projetado para grades de latitude e longitude, 
foi adaptado à esfera cubada para melhorar a escalabilidade em supercomputadores massivamente paralelos. 
Desde então, o FV3 tornou-se popular para modelagem atmosférica global e serve como núcleo dinâmico para muitos modelos globais.
Além disso, em 2019, o FV3 foi selecionado como o núcleo dinâmico oficial para
o novo Sistema Global de Previsão do Serviço Nacional de Meteorologia dos EUA, substituindo o modelo espectral. 
O FV3 emprega uma esfera cubada com discretização de volume finitos para resolver equações que modelam a dinâmica atmosférica.
A abordagem de volume finitos do FV3 para resolver a dinâmica horizontal consiste na aplicação de fluxos de advecção para diversas variáveis;
assim, o esquema de advecção desempenha um papel fundamental no modelo.
Portanto, nesta tese, propomos investigar os detalhes do esquema de advecção do FV3.
Conseguimos sugerir pequenas modificações no esquema de advecção do FV3 que melhoraram significativamente a advecção com apenas um pequeno esforço computacional adicional.
Realizamos várias simulações numéricas usando as equações de advecção e águas rasas.
Como o esquema de advecção do FV3 consiste na combinação de operadores de fluxo de volume finitos 1D,
nossas melhorias foram obtidas ao melhorar o cálculo do ponto de partida para os operadores de fluxo 1D e
modificar a forma como o termo métrico da esfera cubada é tratado ao calcular os fluxos 1D.
Através de simulações, demonstramos que o esquema de advecção atual do FV3 é apenas de primeira ordem para ventos divergentes,
enquanto nosso esquema é sempre de segunda ordem.
Uma grande dificuldade em trabalhar na esfera cubada é lidar com a descontinuidade das coordenadas ao longo das faces do cubo,
o que pode levar a erros maiores nessas regiões.
No entanto, demonstramos através de simulações numéricas que o esquema de advecção proposto apresenta uma sensibilidade reduzida aos cantos do cubo.
}

\abstract{
The dynamical core FV3 from GFDL-NOAA-USA, originally designed for latitude-longitude grids, 
was adapted to the cubed-sphere to improve scalability on massively parallel supercomputers.
FV3 has become popular for global atmospheric modeling and serves as the dynamical core for many models worldwide.
Additionally, in 2019, FV3 was selected as the official dynamical core for the new Global Forecast System of the National Weather Service in the USA,
replacing the spectral model. FV3 employs a cubed-sphere with finite-volume discretization to solve equations modeling atmospheric dynamics.
The finite-volume approach of FV3 for solving horizontal dynamics involves applying advection fluxes for different variables; thus, the advection scheme plays a key role in the model. 
Therefore, in this thesis, we propose to investigate the details of the advection scheme of FV3.
We were able to suggest minor modifications to FV3 advection scheme that significantly improved advection with only a small extra computational effort.
We conducted numerous numerical simulations using the advection and shallow-water equations.
As the FV3 advection scheme consists of combining 1D finite-volume flux operators,
our improvements were obtained by improving the departure point computation for the 1D flux operators
and modifying the way the metric term of the cubed-sphere is treated when computing the 1D fluxes.
Through simulations, we demonstrate that the current FV3 advection scheme is only first-order for divergent winds, while our scheme is always second-order.
One major difficulty in working on the cubed-sphere is handling coordinate discontinuity along the cube faces, which may lead to larger errors in these regions. 
However, we demonstrate through numerical simulations that the proposed advection scheme exhibits reduced sensitivity to the cube corners.
}
