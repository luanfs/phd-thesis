\chapter{Conclusions}
\label{chp-conclusions}
The FV3 dynamical core has become very popular in the atmospheric modeling community.
It has received more attention, especially after being adopted as the new Global Forecast System (GFS) of the USA.
The objective of this thesis is to investigate all the details of the FV3 advection scheme,
as the horizontal dynamics of FV3 are solved using only advection finite-volume fluxes, thus playing a key role in FV3.
The major outcome of this thesis is a more accurate 2D advection scheme than the current FV3 advection scheme, as demonstrated in numerous numerical simulations.

The motivation for the new advection scheme method started in Chapter \ref{chp-1d-fv}, 
where we provide a proof that the time-averaged fluxes of 1D finite-volume methods for the 1D advection equation require two tasks: 
a departure point calculation and the reconstruction of the solution using the average values.
The average values are reconstructed using PPM, just as in FV3.
We note that the FV3 scheme uses a first-order departure point scheme. 
We demonstrate how we can compute the departure point using a second-order Runge-Kutta scheme, which provides us with second-order accuracy.
Then, we observed in numerical tests that this scheme improves accuracy significantly,  even with monotonicity constraints,
with only minor extra computational efforts.
Namely, we only need to perform one linear interpolation for each edge per time step.
We could only observe this improvement because we considered a  variable wind test for the 1D linear advection equation. 
Most 1D tests in the literature use a constant wind, and therefore a first-order departure point is exact.
This oversimplifies matters, as the departure point issue does not arise.

Next, we moved to the 2D advection equation on the plane in Chapter \ref{chp-2d-fv}.
The 2D advection scheme of FV3 consists of combining 1D flux PPM operators.
This combination is made in such a way that when the scalar field is constant and the wind is divergence-free, the scheme is exact.
We observed that this scheme is second-order for divergence-free winds;
however, in a numerical simulation for a divergent wind, we showed that the FV3 scheme is only first-order.
We then demonstrated how we may modify the FV3 scheme to achieve second-order accuracy for both divergence-free and divergent winds. 
This modification involves a slight change of the inner 1D flux operators,
as well as the incorporation of the second-order departure point scheme outlined in Chapter \ref{chp-1d-fv}.
Although we lost the preservation of a constant scalar field, the error is only second-order accurate for this case.
We show that the new scheme has slightly improved performance for divergence-free winds, but for divergent winds, the results are significantly better.
Then, we proposed a scheme that is second-order in general, whereas the FV3 scheme is second-order only for divergence-free winds.

Following that, our next objective was to study the advection equation on the sphere. 
We provided all the tools needed in Chapter \ref{chp-cs-grids}, 
where we presented the duo-grid, which consists of extending the gridlines of the cubed sphere mapping. 
We may then use 1D Lagrangian interpolation to fill the ghost cell values.
In this Chapter, we also introduce the equiangular cubed sphere, which is the most uniform cubed-sphere available in the literature.
We also introduce the equi-edge grid, which is less uniform than the equiangular grid in general but offers more uniformity near to the cube edges,
aiming to avoid grid imprinting in these regions.
Additionally, we show that Lagrange interpolation based on geodesic distances 
is much less accurate than using Lagrange interpolation based on local cubed sphere mapping coordinates.

Using all these tools, we could solve the advection equation on the sphere in Chapter \ref{chp-cs-fv}.
We presented the advection scheme of FV3 and introduced our new scheme. We observed that our new scheme
requires a different treatment of the metric terms in the flux 1D computation.
Essentially, we extended the results from the plane to the cubed-sphere.
Our scheme is second-order accurate on the cubed-sphere, 
while the FV3 scheme is second-order only for divergence-free winds and first-order for divergent winds.
This was observed on both equi-edge and equiangular grids. Additionally, the equi-edge grids yielded smaller errors.
Furthermore, the new scheme was slightly less sensitive to the corners, as it did not show slightly smaller errors at these locations. 
This was observed in tests where we evaluated a Gaussian hill, cosine bell, and a cylinder over the corners.

As an application of our new scheme, in Chapter \ref{chp-cs-swm}, we provide a comprehensive description of the FV3 shallow-water solver.
This solver utilizes the shallow-water equations in vector form and employs only advective finite volume fluxes
to compute the fluid depth, kinetic energy, and absolute vorticity fluxes. 
We consider our scheme only to solve the continuity equation and the absolute vorticity fluxes.
We observe that for the geostrophic balanced flow test case, our scheme helps to reduce the maximum error slightly.
Additionally, we analyze the total runtime for each method at different grid resolutions, and we conclude that our scheme adds a very small extra cost.
{Another major conclusion of this chapter is that the equiangular grid is less susceptible to numerical instability,
being able to perform better without the requirement of divergence damping.
This was observed for both advection schemes.
Finally, for the Rossby-Haurwitz wave test, the new scheme maintains the wave shape for 55-60 days, which is similar to the spectral model. The FV3 scheme, on the other hand, preserves the wave for approximately 95-100 days.}

\section{Future work ideas}
A possible extension of this work would be to consider the non-hydrostatic solver of FV3 to assess the new advection scheme's ability 
in three dimensions and conduct more realistic simulations, as we have only analyzed idealized test cases in this work.
Furthermore, we could also investigate the new scheme's performance on the stretched grids available in FV3.

Our modified advection scheme is slightly better for divergence-free flows in some situations, but for divergent winds, our new scheme is significantly better. 
We expect that our scheme would perform similarly to the current FV3 method in processes where there is small divergence/convergence of the wind, 
and it should yield better results for processes where divergence plays a key role.

Horizontal wind divergence plays a pivotal role in many phenomena on the atmosphere such as in tropical cyclones, hurricanes and in the Intertropical Convergence Zone \citep{holton:2012}.
For example, hurricanes are fueled by strong horizontal wind convergence at the Earth's surface, with strong horizontal wind divergence occurring at high altitudes.
We could perform a study based on \citet{gao:2021}, where the authors investigated the impact of using different PPM schemes of FV3 on hurricane intensity prediction.
Their study highlights how modifying the advection scheme may improve hurricane intensity prediction and affect the eyewall convection location.
Therefore, we could use our advection scheme in these simulations and observe its effect on hurricane intensity prediction.
