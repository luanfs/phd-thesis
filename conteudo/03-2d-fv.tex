\chapter{Two-dimensional finite-volume methods}
\label{chp-2d-fv}
In Chapter \ref{chp-1d-fv}, we addressed the problem of solving the one-dimensional
linear advection equation using the finite-volume method based on PPM. In this chapter, our focus shifts to solving the two-dimensional 
linear advection equation using the finite-volume method. This step is crucial in our work since, as we will explore in Chapter 
\ref{chp-cs-fv}, solving the linear advection equation on the cubed-sphere relies on solving two-dimensional linear advection equations 
at each cube face, with interpolation between adjacent panels.

A natural approach to develop a finite-volume method for the two-dimensional linear advection equation would involve extending PPM to 
two dimensions. Indeed, \citet{rancic:1992} proposed a piecewise bi-parabolic extension of PPM using a semi-Lagrangian temporal 
discretization. 
Further, this type of method can be extended to the cubed-sphere \citep{lauritzen:2010}.
However, this method suffers from a significant drawback—its computationally expensive nature. As a popular alternative, 
dimension-splitting methods are often used, which replace the two-dimensional problem with a sequence of one-dimensional problems. For 
example, we can solve the two-dimensional linear advection equation by solving a series of one-dimensional linear advection equations 
using the PPM from Chapter \ref{chp-1d-fv}. Moreover, in principle, we can employ any numerical method that solves the one-dimensional 
linear advection equation.

A comparison between two-dimensional and dimension-splitting semi-Lagrangian schemes on a plane was investigated by \citet{chen:2017}, 
utilizing the PPM as the one-dimensional solver and distorted two-dimensional grids. Their main conclusion was that dimension-splitting 
schemes are more sensitive to grid distortions, but they are computationally cheaper and more accurate than two-dimensional methods, 
particularly when dealing with large CFL numbers.

The primary objective of this chapter is to provide a comprehensive explanation of the dimension splitting method proposed by 
\citet{lin:1996}. This method is currently utilized in the FV3 dynamical core and is applied to the two-dimensional linear advection 
equation using the one-dimensional finite-volume schemes described in Chapter \ref{chp-1d-fv}.
Similar to Chapter \ref{chp-1d-fv}, we start this chapter with a review of the integral form of the two-dimensional advection 
equation in Section \ref{sec-adv2d}. In Section \ref{sec:fv-2d}, we establish the framework for general two-dimensional finite-volume 
schemes. The dimension splitting method is presented in Section \ref{sec-dsplit}, and we showcase numerical experiments in Section 
\ref{sec-ds-exp}.
\section{Two-dimensional advection equation in integral form}
\label{sec-adv2d}

