\chapter{Cubed-sphere finite-volume methods}
\label{chp-cs-fv}
Now that we have described in Chapter \ref{chp-cs-grids} how we can obtain the ghost cell values of each panel on the cubed-sphere using Lagrange interpolation, we are ready to apply the dimension-splitting methods presented in Chapter \ref{chp-2d-fv} to solve the advection equation on the cubed-sphere.
One significant difference is that we have the metric term, which is not present in the plane simulations.
Additionally, when employing ghost cell layers using the duo-grid, the flux at the cube edges is computed twice, 
requiring the averaging of fluxes at the edges to ensure a unique value in order to achieve mass conservation.


This Chapter is organized as follows: Section \ref{chp-cs-adv} introduces the advection equation on the cubed-sphere.
Section \ref{chp-cs-fvcs} presents its finite-volume discretization with a focus on the extension of dimension splitting (Section \ref{sec-csdsplit}) 
as presented in Section \ref{sec-dsplit}.
Numerical experiments are presented in Section \ref{chp-cs-numexpadv}, where we use dimension splitting to assess its
accuracy in computing the divergence of a given vector field to check its numerical consistency, as well as to solve the advection equation.
In particular, we explore different treatments for the cube edges. Section \ref{chp-cs-conc} presents the final thoughts.

\section{Cubed-sphere advection equation in integral form}
\label{chp-cs-adv}
Given a tangent velocity field $\boldsymbol{u}$ on the sphere, we denote its
contravariant components by $\mathfrak{u}$ and $\mathfrak{v}$.
We shall use all the notations introduced in Section \ref{cs-notation}.
The advection equation on panel the $p$ of the cubed-sphere with initial condition $q_0$ is given by:
\begin{equation}
	\begin{cases}
		\label{eq1-adv-cs}
		\bigg[{\partial}_t{q}+
		\frac{1}{\sqrt{\mathfrak{g}}}\bigg(
		{\partial}_x{(\mathfrak{u} q \sqrt{\mathfrak{g}})}+
		{\partial}_y{(\mathfrak{v} q \sqrt{\mathfrak{g}})}
		\bigg)\bigg](x,y,p,t)
		= 0,\\
		q(x,y,p,0) = q_0(x,y,p),
	\end{cases}
\end{equation}
$\forall (x,y) \in \Omega := [-\alpha,\alpha]^2$, $t\in[0,T]$.
We denote by $\nabla \cdot (q\boldsymbol{u})$ the divergence operator:
\begin{equation}
	\label{advcs:eqdiv}
	\nabla \cdot (q\boldsymbol{u})(x, y, p, t) =  \frac{1}{\sqrt{\mathfrak{g}}}
	[{\partial_x (\mathfrak{u}q\sqrt{\mathfrak{g}})} + {\partial_y (\mathfrak{v}q\sqrt{\mathfrak{g}})}](x, y, p, t).
\end{equation}
We recall that we say the $\boldsymbol{u}$ is \textbf{non-divergent} if $\nabla \cdot \boldsymbol{u}=0$.
We define the $\mathcal{CS}_N$ grid function $\delta^n$ as
the exact divergence of $q\boldsymbol{u}$ at the cell centers, namely
\begin{equation}
	\label{cs-discrete-div}
	\delta^n_{ijp} = \nabla \cdot (\boldsymbol{u}q)(x_i,y_j,p,t^n).
\end{equation}
In this Chapter, it shall be useful to define the average value of $q\sqrt{\mathfrak{g}}$ on the 2D coordinates as:
\begin{equation}
\label{cs-q-av}
\overline{(\sqrt{\mathfrak{g}}q)}_{ijp}(t) = \frac{1}{\Delta x \Delta y}
\int_{x_{i-\frac{1}{2}}}^{x_{i+\frac{1}{2}}}
\int_{y_{j-\frac{1}{2}}}^{y_{j+\frac{1}{2}}}  q(x,y,p,t) {\sqrt{\mathfrak{g}}(x,y)}\,dx \,dy.
\end{equation}
This average value simplifies the deduction of finite-volume method on the cubed-sphere instead of using the spherical average values (Equation \eqref{qav-sphere}).
Since the metric term does not depend on $t$, we may rewrite Equation \eqref{eq1-adv-cs} as
\begin{equation}
	\label{eq2-adv-cs}
	\bigg[{\partial}_t{(q \sqrt{\mathfrak{g}})}+
	{\partial}_x{(\mathfrak{u}q \sqrt{\mathfrak{g}})}+
	{\partial}_y{(\mathfrak{v}q \sqrt{\mathfrak{g}})}
	\bigg](x,y,p,t)
	= 0.
\end{equation}
Therefore, as in Problem \eqref{chp-adv2d-sec2-prob1}, the integral form of Equation \eqref{eq1-adv-cs}
is stated in Problem \eqref{chp5-prob1}.
\begin{prob}
	\label{chp5-prob1}
	Given an initial condition ${q}_0$ and
	a velocity on the sphere $\boldsymbol{u}$, with contravariant components $(\mathfrak{u},\mathfrak{v})$ on the cubed-sphere coordinate system,
	we would like to find a weak solution ${q}$
	of the cubed-sphere advection equation in its integral form:
	\begin{align*}
		\int_{x_1}^{x_2} \int_{y_1}^{y_2}
		{(q\sqrt{\mathfrak{g}})}(x, y, p, t) \,dx \,dy = &\int_{x_1}^{x_2} \int_{y_1}^{y_2}
		{(q\sqrt{\mathfrak{g}})}(x, y, p, t) \,dx \,dy \\ \nonumber
		&-\int_{t_1}^{t_2} \int_{y_1}^{y_2} \bigg({(\mathfrak{u}q\sqrt{\mathfrak{g}})}(x_2, y, t)
		-{(\mathfrak{u}q\sqrt{\mathfrak{g}})}(x_1, y, t) \bigg) \,dy \,dt\\ \nonumber
		&-\int_{t_1}^{t_2} \int_{x_1}^{x_2} \bigg({(\mathfrak{v}q\sqrt{\mathfrak{g}})}(x, y_2, t)
		-{(\mathfrak{v}q\sqrt{\mathfrak{g}})}(x, y_1, t) \bigg) \,dx \,dt.
	\end{align*}
	$\forall [x_1, x_2]\times [y_1, y_2] \times[t_1, t_2] \subset \Omega \times[0,T]$, and
	$q(x,y,p,0)=q_0(x,y,p)$.
\end{prob}
Similarly to Section \ref{chp-adv2d-sec1}, Equation \eqref{eq1-adv-cs} and Problem \eqref{chp5-prob1} are equivalent
when ${q}, \boldsymbol{u} \in \mathcal{C}^1(\mathbb{S}^2_R)$.
For Problem \ref{chp5-prob1}, the total mass in $\mathbb{S}^2_R$ is defined by: 
\begin{equation}
	{M}_{\mathbb{S}^2_R}(t) = \sum_{p=1}^6 \int_{\Omega} {(q\sqrt{\mathfrak{g}})}(x,y,p,t) \,dx \,dy , \quad \forall t \in [0,T],
\end{equation}
and is conserved within time: 
\begin{equation}
	{M}_{\mathbb{S}^2_R}(t) = {M}_{\mathbb{S}^2_R}(0), \quad \forall t \in [0,T].
\end{equation}
We define a discretized version of Problem \eqref{chp5-prob1} as Problem \eqref{chp5-prob2}.
\begin{prob}
	\label{chp5-prob2}
	Assume the framework of Problem \ref{chp5-prob1}
	and consider a $(\Delta x, \Delta y, \Delta t, \lambda)$-discretization of $\Omega\times [0,T]$, with $\Delta x= \Delta y$.
	Since we are in the framework of Problem \ref{chp5-prob1}, it follows that:
	\begin{align*}
		\overline{(\sqrt{\mathfrak{g}}q)}_{ijp}(t_{n+1})  = \overline{(\sqrt{\mathfrak{g}}q)}_{ijp}(t_{n})
		&- { \lambda}
		\delta _x \bigg( \frac{1}{\Delta t \Delta y}
		\int_{t^n}^{t^{n+1}} \int_{y_{j-\frac{1}{2}}}^{y_{j+\frac{1}{2}}} 
		{(\mathfrak{u}q\sqrt{\mathfrak{g}})}(x_{i}, y, p, t)
		\,dy \,dt \bigg) \\ \nonumber
		&- {\lambda}
		\delta _y \bigg( \frac{1}{\Delta t \Delta x}
		\int_{t^n}^{t^{n+1}} \int_{x_{i-\frac{1}{2}}}^{x_{i+\frac{1}{2}}} 
		{(\mathfrak{v}q\sqrt{\mathfrak{g}})}(x, y_{j}, p, t)
		\,dx \,dt \bigg),
	\end{align*}
	where
	\begin{equation}
		\overline{(\sqrt{\mathfrak{g}}q)}_{ijp}(t) = \frac{1}{\Delta x \Delta y}
		\int_{x_{i-\frac{1}{2}}}^{x_{i+\frac{1}{2}}} 
		\int_{y_{j-\frac{1}{2}}}^{y_{j+\frac{1}{2}}} {(q\sqrt{\mathfrak{g}})}(x,y,p,t) \,dx \,dy.
	\end{equation}
	Our problem now consists of finding the values ${Q}_{ijp}(t_{n})$, 
	$\forall i = 1, \ldots, N$, $\forall j = 1, \ldots, M$, $\forall n = 0, \ldots, N_T-1$,
	given the initial values ${(\sqrt{\mathfrak{g}}q)}_{ijp}(0)$, $\forall i = 1, \ldots N$, $\forall j = 1, \ldots, M$.
	In other words, we aim to find the average values of ${(\sqrt{\mathfrak{g}}q)}_{ijp}$ in each control volume $\Omega_{ijp}$ at the specified time instances.
\end{prob}
It is important to note that no approximations have been made in Problems \eqref{chp5-prob1} and \eqref{chp5-prob2}. 
%In practice, the term $\frac{\Delta x  \Delta y}{|\Omega_{ijp}|}$ is estimated using the second-order
%formula obtained by applying the midpoint rule in Equation \eqref{chp4-area}:
%\begin{equation}
%	\frac{\Delta x  \Delta y}{|\Omega_{ijp}|}= \frac{1}{\sqrt{\mathfrak{g}}_{ijp}} + O(\Delta x^2).
%\end{equation}
\section{Finite-volume on the cubed-sphere approach}
\label{chp-cs-fvcs}
We are ready to introduce the finite-volume scheme on the cubed-sphere (CS-FV).
A CS-FV scheme problem as follows in Problem \ref{chp5-prob3}.
Before that, we consider the following approximation, which follows from the midpoint rule (Theorem \ref{prop-bound-midpoint2d}):
\begin{equation}
	\label{midpoint-approx}
	\overline{(\sqrt{\mathfrak{g}}q)}_{ijp}(t)  = \sqrt{\mathfrak{g}_{ij}} {q}_{ijp}(t) +\mathcal{O}(\Delta x^2).
\end{equation}
We use this approximation in Problem{chp5-prob2} and we obtain the following CS-FV scheme:
\begin{prob}[CS-FV scheme]
	\label{chp5-prob3}
	Assume the framework defined in Problem \ref{chp5-prob2}.
	The finite-volume approach of Problem \ref{chp5-prob1}
	consists of a finding a scheme of the form:
	\begin{align}
		\label{chp5-csfv}
		{q}_{ijp}^{n+1} =  {q}_{ijp}^{n} - \frac{\lambda}{\sqrt{\mathfrak{g}}_{ij}} \delta_i {F}_{ijp}^{n}
		-  \frac{\lambda}{\sqrt{\mathfrak{g}}_{ij}}  \delta_j {G}_{ijp}^{n},
		\\ \nonumber \quad \forall i = 1, \ldots, N, \quad \forall j = 1, \ldots, M, \quad p =1, \ldots, 6,
		\quad \forall n = 0, \ldots, N_T-1,
	\end{align}
	where $ \delta_i F_{ijp}^n =
	{F}_{i+\frac{1}{2},j,p}^{n} 
	- {F}_{i-\frac{1}{2},j,p}^{n}$,
	$ \delta_j G_{ijp}^n =
	{G}_{i,j+\frac{1}{2},p}^{n} 
	- {G}_{i,j-\frac{1}{2},p}^{n}$ 
	and ${q}^{n}\in \mathcal{CS}_N$ is intended to be an approximation
	of ${q}(t_{n})\in \mathcal{CS}_N$ in some sense. We define
	${q}_{ijp}^{0} = {q}^0_{ijp}$.
	
	The term ${F}_{i+\frac{1}{2}, j, p}^{n}$ is known as numerical flux in the 
	$x$ direction and it approximates
	$\frac{1}{\Delta t \Delta y }\int_{t_n}^{t_{n+1}} 
	\int_{y_{j-\frac{1}{2}}}^{y_{j+\frac{1}{2}}} 
	(\mathfrak{u}q\sqrt{\mathfrak{g}})(x_{i+\frac{1}{2}}, y, p, t) \,dy \,dt $,
	$\forall i = 0, 1, \ldots, N$, and 
	${G}_{i, j+\frac{1}{2}, p}^{n}$ is known as numerical flux in the 
	$y$ direction and it approximates
	$\frac{1}{\Delta t\Delta x}\int_{t_n}^{t_{n+1}}  
	\int_{x_{i-\frac{1}{2}}}^{x_{i+\frac{1}{2}}}
	(\mathfrak{v}q\sqrt{\mathfrak{g}})(x, y_{j+\frac{1}{2}}, p, t) \,dx \,dt $,
	$\forall j = 0, 1, \ldots, M$,
	or, in other words, they estimate the time-averaged
	fluxes at the control volume $\Omega_{ijp}$ boundaries.
\end{prob}
\begin{remark}
	For Problem \ref{chp5-prob3}, we define the CFL number in the $x$ and $y$ direction
	by $\max \{{|\mathfrak{u}_{i+\frac{1}{2},j,p}^n}|\}\frac{\Delta t}{\Delta x}$ and 
	$\max \{ {|\mathfrak{v}_{i,j+\frac{1}{2},p}^n}|\}\frac{\Delta t}{\Delta y}$, respectively.
	The CFL number is maximum between these numbers and we say that the CFL condition is
	satisfied if the CFL number is less than one. 
\end{remark}
As we mentioned in Problem \ref{chp5-prob3}, the initial condition may be assumed as $q_{ijp}^0$ or $Q_{ijp}(0)$.
We are going to assume  $q_{ijp}^0$ as initial data to avoid the computation of integrals.
Furthermore, the errors will be calculated using the values $q_{ijp}^n$ instead of $Q_{ijp}(t_n)$.
As in Section \ref{sec:fv-2d} this approximation leads to a second-order error.


\section{Dimension splitting}
\label{sec-csdsplit}
In this Section, we will utilize the dimension splitting method described in Section \ref{sec-dsplit} to obtain a CS-FV scheme.
To facilitate notation, we shall omit the index $p$ whenever it may appear in this Section, as what is described here does not depend on $p$.
Also, the ghost cell values are assumed to be filled using the duo-grid interpolation.
\subsection{PPM and the metric term}
Recall that the dimension splitting technique requires the numerical solution of advection in the $x$ and $y$ directions for separability.
For instance, in the case of the advection equation on the cubed-sphere (Equation \eqref{eq2-adv-cs}), we need to solve the following equations in the $x$ direction:
\begin{equation}
	\label{1d-avd-eq}
		[{\partial_t (\sqrt{\mathfrak{g}} {q^x})} + {\partial_x (\mathfrak{u}\sqrt{\mathfrak{g}}{q^x}})](x, y_j, p, t),
\end{equation}
for $j=-\nu+1, \ldots, N+\nu$, at certain time levels $t^n$, $n=1, \ldots, N_T$ (Section \ref{lit-trotter-sp}).
We are particularly interested in approximating $q^{x,n+1}_{ij}$ for $i=1, \ldots, N$, which represents the values of $q^x$ at the cell centroids.
This involves providing an approximation of the solution to Equation \eqref{eq2-adv-cs}, 
denoted as $q^n_{ij}$, serving as initial data at time level $n$, specifically $q^{x,n}_{ij}=q^n_{ij}$.

Considering the midpoint approximation of the average value (Equation \eqref{midpoint-approx}), we approximate
the solution of the desired problem using an general 1D FV-SL scheme as discussed in Section \ref{chp-adv1d-sec2-fvsl}:
\begin{equation}
	\label{1d-qx}
	q^{x,n+1}_{ij} = q^{n}_{ij}  - \frac{\Delta t}{\sqrt{\mathfrak{g}_{ij}}\Delta x}
	\bigg[{F}_{i+\frac{1}{2}}
	\big({q}^{n}; \tilde{c}^{x,n}_{i+\frac{1}{2},j}\big) - 
	{F}_{i-\frac{1}{2},j}
	\big({q}^{n}; \tilde{c}^{x,n}_{i-\frac{1}{2},j}\big)\bigg],
\end{equation}
for  $j=-\nu+1, \ldots, N+\nu$ and $i=1, \ldots, N$, where
\begin{equation}
	\label{mt0-flux}
	{F}_{i\pm\frac{1}{2},j} = \frac{1}{\Delta t}\int_{ \tilde{x}_{i\pm\frac{1}{2},j}^n}^{x_{i\pm\frac{1}{2}}}(\widetilde{\sqrt{\mathfrak{g}}{q}})_j(x, t^n) \,dx,
\end{equation}
$\tilde{x}_{i\pm\frac{1}{2},j}^n$ is an estimate of the departure point in $x$ direction using the time-averaged
CFL number $\tilde{c}^{x,n}_{i+\frac{1}{2},j}$ (Section \ref{chp-adv1d-sec-dp}),
and $\widetilde{\sqrt{\mathfrak{g}}{q}}_j$ is a PPM reconstruction (or any other reconstruction) of $\sqrt{\mathfrak{g}}{q}$ (Section \ref{chp-adv1d-sec-recon})
in the $x$ direction ($j$ is fixed).

It is also possible to compute the PPM reconstruction in terms only of $q$, ignoring the metric term $\sqrt{\mathfrak{g}}$.
In other words, we may assume that the metric is constant on each integration domain, which leads to the following first-order error:
\begin{equation}
\int_{ \tilde{x}_{i\pm\frac{1}{2},j}^n}^{x_{i\pm\frac{1}{2}}}(\widetilde{\sqrt{\mathfrak{g}}{q}})(x, t^n) \,dx
= \sqrt{\mathfrak{g}}_{i\pm\frac{1}{2},j}\int_{  \tilde{x}_{i\pm\frac{1}{2},j}^n}^{x_{i\pm\frac{1}{2}}}\widetilde{q}(x, t^n) \,dx + \mathcal{O}(\Delta x).
\end{equation}
In this case, the flux reads:
\begin{equation}
	\label{mt1-flux}
	{F}_{i\pm\frac{1}{2},j} =  \frac{\sqrt{\mathfrak{g}}_{i\pm\frac{1}{2},j}}{\Delta t}\int_{  \tilde{x}_{i\pm\frac{1}{2},j}^n}^{x_{i\pm\frac{1}{2}}}\widetilde{q}(x, t^n)  \,dx.
\end{equation}
Then, in this case, we perform the PPM flux for the grid function $q^n$.
When we compute the flux using Equation \eqref{mt0-flux}, we denote this by \textbf{mt0};
when using Equation \eqref{mt1-flux}, we denote this by \textbf{mt1}.

The works of \citet{lin:2004} and \citet{putman:2007} use the mt1 method, which is currently employed in FV3.
This process, although it introduces a first-order error, 
significantly simplifies the elimination of the splitting error that arises when $q_{ij} = \overline{q}$, for a constant $\overline{q}$, 
and when the wind is divergence-free.
This occurs because when we use mt1, we have
\begin{equation}
	\label{mt1-flux2}
	{F}_{i\pm\frac{1}{2},j} =  \overline{q}\frac{\sqrt{\mathfrak{g}}_{i\pm\frac{1}{2},j}}{\Delta t}
	\delta_i c^{x,n}_{ij},
\end{equation}
assuming that the departure point is computed using the DP1 method for the departure point calculation (as discussed in Section \ref{sp-error}).
The property from Equation \eqref{mt1-flux2} does not occur for mt0.

For a CS-grid function $\psi \in \mathcal{CS}_N$
we introduce the following PPM flux in the $x$ direction (recall Equation \eqref{chp-sec-flux:numerical-flux8})
\begin{align}
	%	\label{chp3-flux-xdir}
	\mathfrak{F}_{i+\frac{1}{2},j}^{PPM,x} [{{\psi}^n;\tilde{c}^{x,n}_{i+\frac{1}{2},j}}]= %\tilde{u}^{n}_{i+\frac{1}{2},j}\times
	\begin{cases}
		{\psi}_{i-1,j}^{n}+(1-\tilde{c}_{i+\frac{1}{2}}^{x,n})
		(b^L_{i,j}-\tilde{c}_{i+\frac{1}{2},j}^{x,n})
		(b^L_{i,j}+b^R_{ij}),
		\quad &\text{if} \quad \tilde{c}_{i+\frac{1}{2},j}^{x,n}>0,\\
		{\psi}_{ij}^{n}+(1+\tilde{c}_{i+\frac{1}{2},j}^{x,n})
		(b^L_{i+1,j}+\tilde{c}_{i+\frac{1}{2},j}^{x,n})
		(b^L_{i+1,j}+b^R_{i+1,j}),
		\quad &\text{if} \quad \tilde{c}_{i+\frac{1}{2},j}^{x,n}\leq0,
	\end{cases}
\end{align}
for each $j=-\nu+1, \ldots, N+\nu$ and $i=1, \ldots, N$, and
where the PPM perturbation values $b^L$ and $b^R$ values are computed using hord0 (Section \ref{chp-adv1d-sec-hord0}) or hord8 (Section \ref{chp-adv1d-sec-hord8})

Therefore, we may rewrite Equation \eqref{1d-qx} as
\begin{equation}
q^{x,n+1}_{ij} = q^{n}_{ij} + \mathbf{F}_{ij}[{q^n,\tilde{c}^{x,n}}],
\end{equation}
for $i=1, \ldots, N$, $j=-\nu+1, \ldots, M + \nu$, and where
\begin{align*}
	\mathbf{F}_{ij}[{q^n,\tilde{c}^{x,n}}] = 
	-\frac{1}{|\hat{\Omega}_{ij}|}
	\bigg(\mathcal{A}_{i+\frac{1}{2},j}^{x} \mathfrak{F}_{i+\frac{1}{2},j}^{x}[q^n_{\times,j},\tilde{c}^{x,n}]-
	\mathcal{A}_{i-\frac{1}{2},j}^{x} \mathfrak{F}_{i-\frac{1}{2},j}^{x}[q^n_{\times,j},\tilde{c}^{x,n}] \bigg),
\end{align*}
recalling the term $|\hat{\Omega}_{ij}|$ from defined Equation \eqref{chp4-area},
and following the discussion on the metric term, we have the coefficients
\begin{align}
	\mathcal{A}_{i+\frac{1}{2},j}^x= 
	\begin{cases}
		\hat{\delta} x_{i+\frac{1}{2},j}  \hat{\delta} y_{i+\frac{1}{2},j} 
		\sin{\alpha_{i+\frac{1}{2},j}}
		{\tilde{c}}_{i+\frac{1}{2},j}^{x,n},
		\quad &\text{for mt0},\\
		{\Delta x}{\Delta y}{\tilde{c}}_{i+\frac{1}{2},j}^{x,n}
		\quad &\text{for mt1},
	\end{cases}
\end{align}
where we have made use of Equation \eqref{distcube2} and Equation \eqref{mt-sina}, 
and the PPM fluxes are
\begin{align}
	\mathfrak{F}_{i+\frac{1}{2},j}^x [{{q}^n;\tilde{c}^{x,n}_{i,j+\frac{1}{2}}}] = 
	\begin{cases}
		\mathfrak{F}_{i+\frac{1}{2},j}^{PPM,x}[{{\sqrt{g}q}^n;\tilde{c}^{x,n}_{i+\frac{1}{2},j}}]
		\quad &\text{for mt0},\\
		\mathfrak{F}_{i+\frac{1}{2},j}^{PPM,x}[{{q}^n;\tilde{c}^{x,n}_{i+\frac{1}{2},j}}]
		\quad &\text{for mt1}.
	\end{cases}
\end{align}
Similarly, we may derive a scheme to solve Equation \eqref{eq2-adv-cs} in the $y$ direction as
\begin{equation}
	q^{x,n+1}_{ij} = q^{n}_{ij} + \mathbf{G}_{ij}[{q^n,\tilde{c}^{x,n}}],
\end{equation}
for $i=-\nu+1, \ldots, N + \nu$  $j=1, \ldots, N$.
\begin{align*}
	\mathbf{G}_{ij}[{q^n,\tilde{c}^{y,n}}] = 
	-\frac{1}{|\hat{\Omega}_{ij}|}
	\bigg(\mathcal{A}_{i,j+\frac{1}{2}}^{y} \mathfrak{F}_{i,j+\frac{1}{2}}^{y}[q^n_{i,\times},\tilde{c}^{y,n}]-
	\mathcal{A}_{i,j-\frac{1}{2}}^{y} \mathfrak{F}_{i,j-\frac{1}{2}}^{y}[q^n_{i,\times},\tilde{c}^{y,n}] \bigg),
\end{align*}
and
\begin{align}
	\mathcal{A}_{i,j+\frac{1}{2}}^y= 
	\begin{cases}
		\hat{\delta} x_{i,j+\frac{1}{2}}  \hat{\delta} y_{i,j+\frac{1}{2}} 
		\sin{\alpha_{i,j+\frac{1}{2}}}
		{\tilde{c}}_{i,j+\frac{1}{2}}^{y,n},
		\quad &\text{for mt0},\\
		{\Delta x}{\Delta y}{\tilde{c}}_{i,j+\frac{1}{2}}^{y,n}
		\quad &\text{for mt1}
	\end{cases}
\end{align}
and the PPM fluxes are
\begin{align}
	\mathfrak{F}_{i,j+\frac{1}{2}} ({{q}^n;\tilde{c}^{y,n}_{i,j+\frac{1}{2}}})= 
	\begin{cases}
		\mathfrak{F}_{i,j+\frac{1}{2}}^{PPM,y}({{\sqrt{g}q}^n;\tilde{c}^{y,n}_{i,j+\frac{1}{2}}})
		\quad &\text{for mt0},\\
		\mathfrak{F}_{i+\frac{1}{2},j}^{PPM,y}({{q}^n;\tilde{c}^{y,n}_{i,j+\frac{1}{2}}})
		\quad &\text{for mt1}.
	\end{cases}
\end{align}

We recall the advection equation on a cubed-sphere panel is just an advection on the plane with a metric.
Then, from Section, we 

\begin{align}
	q^{n+1} = q^n &+ \frac{1}{2}\mathbf{F}[q^n,\tilde{c}^{x,n}] + \frac{1}{2}\mathbf{G}[q^n,\tilde{c}^{y,n}]\nonumber \\
	&+\frac{1}{2}\mathbf{F}\bigg[q^n + \mathbf{g}[q^n, \tilde{c}^{y,n}], \tilde{c}^{x,n}\bigg]+
	\frac{1}{2}\mathbf{G}\bigg[q^n + \mathbf{f}[q^n, \tilde{c}^{x,n}], \tilde{c}^{y,n}\bigg],
\end{align}
We start by introducing the CFL number as follows (recall the wind formulation in Section \ref{anexo-sph-ll}):
\begin{align*}
	{c}_{i+\frac{1}{2},j}^{x,n} = {\mathfrak{u}}_{i+\frac{1}{2},j,p}^{x,n}\frac{\Delta t}{\Delta x}
	= {{u}}_{i+\frac{1}{2},j}^{x,n}\frac{\Delta t}{\hat{\delta} x_{i+\frac{1}{2},j}},\\
	{c}_{i,j+\frac{1}{2}}^{y,n} = {\mathfrak{v}}_{i,j+\frac{1}{2},p}^{y,n}\frac{\Delta t}{\Delta y}
	= {{v}}_{i,j+\frac{1}{2}}^{y,n}\frac{\Delta t}{\hat{\delta} y_{i,j+\frac{1}{2}}}.\\
\end{align*}
In FV3, the terms ${\hat{\delta} x_{ij}}$ and ${\hat{\delta} y_{ij}}$ and $|\hat{\Omega}_{ij}|$ (for integers of half integers $i$ and $j$)
are replaced by ${{\delta} x_{ij}}$, ${{\delta} y_{ij}}$ and $|\hat{\Omega}_{ij}|$,
which represent the geodesic distances and areas (Section \ref{cs-geo}).

The discrete divergence is then obtained as:
\begin{equation}
	\label{eqdiv-split}
	\mathbb{D}^n_{ijp} = -\frac{1}{\Delta t}
	\bigg[
	\mathbf{F}_{ijp}\bigg(Q^n + \frac{1}{2}\mathbf{g}(Q^n,\tilde{v}^n), \tilde{u}^n \bigg) 
	+\mathbf{G}_{ijp}\bigg(Q^n + \frac{1}{2}\mathbf{f}(Q^n,\tilde{u}^n), \tilde{v}^n \bigg) \bigg],
\end{equation}
where the inner advective operators $\mathbf{f}$ and $\mathbf{g}$ are given in Table \ref{chp3-tab1}.

Now, our objective is to describe the 1D fluxes ${F}_{i+\frac{1}{2},j,p}^n$ and ${G}_{i,j+\frac{1}{2},p}^n$. 
It is important to note that these fluxes depend on the edge treatment, as well as the computation of stencils
and metric tensor treatment, as we will see in the following section.


\subsection{Flux at edges treatment}
As in Section \ref{sec:fv-2d}  we introduce the notion of discrete divergence,
which allow us to check the consistency of CS-FV schemes.
\begin{definition}[Discrete divergence]
	\label{chp5-def-div}
	For Problem \ref{chp5-prob3}, we define the discrete divergence as a 
	$\mathcal{CS}_N$-grid function $\mathbb{D}^n(Q^n,\mathfrak{u}^n,\mathfrak{v}^n)$
	given by:
	\begin{equation}
		\label{chp5-def-div-eq}
		\mathbb{D}_{ijp}^n(Q^n,\mathfrak{u}^n,\mathfrak{v}^n)=  \frac{1}{\Delta t |\Omega_{ij}|}
		\bigg({\delta_i {F}_{ijp}^{n}} + {\delta_j {G}_{ijp}^{n}}\bigg), 
		\quad i = 1, \ldots, N, \quad j=1, \ldots,M.
	\end{equation}
\end{definition}
With the aid of the discrete divergence, Equation \eqref{chp5-csfv} becomes:
\begin{equation}
	\label{chp5-def-div-eq2}
	Q^{n+1} = Q^n - \Delta t \mathbb{D}^n(Q^n,\mathfrak{u}^n,\mathfrak{v}^n).
\end{equation}
For a CS-FV scheme the discrete total mass at the time-step $n$ is given by
\begin{equation*}
	M^n =\sum_{p=1}^6 \sum_{i,j=1}^N Q_{ijp}^n |\Omega_{ij}|
\end{equation*}
It follows from Equation \eqref{chp5-def-div-eq2} that:
\begin{align*}
	M^{n+1} &= M^n  - \sum_{p=1}^6 \sum_{i,j=1}^N \mathbb{D}_{ijp}^{n} |\Omega_{ij}|.
\end{align*}
Hence, to ensure mass conservation, we must ensure that
\begin{align*}
	\sum_{p=1}^6 \sum_{i,j=1}^N   \mathbb{D}_{ijp}^{n} |\Omega_{ij}| = 0.
\end{align*}
This property is discrete version of
\begin{align*}
	\int_{\mathbb{S}^2_R} \nabla \cdot (\boldsymbol{u}q) \,dS = 0,
\end{align*}
which follows from the divergence theorem and the fact of the sphere has no boundary, where $\,dS$ is the surface measure of the sphere.

When computing the flux, if we ignore the discontinuity in the cubed sphere coordinate system and use values from adjacent panels
(as in the kinked scheme from Chapter \ref{chp-cs-grids}) to compute stencils, we can ensure mass conservation because the
flux at points lying on the cube edge will be the same.
However, if we consider ghost cell layers by extending the gridlines (as in the duo-grid scheme from Chapter \ref{chp-cs-grids}),
the flux is computed twice at points lying on the cube edge.
Therefore, in this case, some modification is needed to ensure mass conservation (Figure \ref{chp5-fluxcube}).
\begin{figure}[!htb]
	\centering
	\includegraphics[width=0.5\linewidth]{flux_interface}
	\caption{Figure that illustrates the flux being computed twice on
		the cube edge, breaking the total mass conservation.
		Figure taken from \citet{ross:2006}.\label{chp5-fluxcube}}
\end{figure}

One common alternative used in the literature to handle the issue of values being defined twice at points on the
cube edges is to simply average the values (as seen in works such as \citet{ross:2006, chen:2008, chen:2021, mouallem:2023}).
When we are using flux averaging, we shall use the label \textbf{mf1}. When no mass fixer is used, we employ the label \textbf{mf0}.

If we choose the ET-DG scheme, we will calculate the required wind values for ghost positions as
described in Section \ref{cs-wind-interp}.
However, this scheme requires modifying the flux at the edges to maintain mass conservation.
To achieve this, we will apply the mass fixer schemes discussed earlier.
Using these flux values, we can compute the discrete divergence (Equation \eqref{eqdiv-split}) for the interior cells and update the solution for the next time step.

On the other hand, if we utilize the ET-PL07 scheme, we will populate the scalar field and velocity field contravariant
components at ghost cell positions with the adjacent panel values, similar to the ET-S72 scheme.
This allows us to compute $\mathbf{F}$ and $\mathbf{G}$ at all necessary positions and we can update the solution.
In the PPM reconstruction, we employ extrapolation near the cubed edges that defines the ET-PL07 scheme.
When combining MT-PL07 with ET-PL07, a first-order error is introduced.
However, we can ensure that the PL07 splitting scheme (Table \ref{chp3-tab1}) eliminates the splitting error when $q$ is constant.

\section{Numerical experiments}
\label{chp-cs-numexpadv}
This Section is dedicated to present the numerical experiments for the
advection equation on the sphere. In Table \ref{chp5-tab1} we present
the initial conditions (IC) and in Table \ref{chp5-tab2} we present
the velocity fields (VF) considered.  