\chapter{One-dimensional finite-volume methods}
\label{chp-1d-fv}

\theoremstyle{plain}
\newtheorem{lema}{Lemma}[chapter]

\theoremstyle{plain}
\newtheorem{prop}{Proposition}[chapter]

\theoremstyle{plain}
\newtheorem{thrm}{Theorem}[chapter]

\theoremstyle{plain}
\newtheorem{remark}{Remark}[chapter]

\theoremstyle{plain}
\newtheorem{corollary}{Corollary}[chapter]

\theoremstyle{plain}
\newtheorem{definition}{Definition}[chapter]


The aim of this chapter is to provide a detailed description of one-dimensional (1D) 
finite-volume (FV) schemes within a Semi-Lagrangian (SL) framework, specifically applied to 
the 1D advection equation.
These schemes are also known as flux-form Semi-Lagrangian schemes, and they allow for time steps beyond the 
Courant-Friedrichs-Lewy (CFL) condition while preserving the total mass.
FV-SL schemes have been explored in the literature since the work of  \citet{leveque:1985},
which extended the finite-volume schemes from \citet{godunov:1959}  to accommodate larger time steps.
This approach has been further investigated in the literature (c.f, e.g. . \citet{lin:1996,leonard:1996}).
We are going to focus on the linear advection equation because in FV3, the horizontal dynamics
are solved by using flux advection operators to compute the fluid density, absolute vorticity, 
and the kinetic energy \citep{lin:1997,putmanthesis:2007, harris:2013, harris:2021}.
The boundary conditions are assumed to be periodic for simplicity.

To introduce the FV-SL schemes, we begin by discretizing the spatial and temporal domains into uniform grids.
Subsequently, the FV-SL schemes involve three steps.
The first step involves computing the departure points of the spatial grid edges.
The second step, known as reconstruction, utilizes the grid cell average values to
determine a piecewise function within each cell. This piecewise function approximates the
values of the advected quantity and ensures the preservation of its local mass within each grid cell.
The third step involves updating the fluxes at the grid edges by integrating the reconstruction function 
over a domain that extends from the departure point of the grid edge to the grid edge itself.

The first step of FV-SL schemes can be accomplished by integrating an ordinary differential
equation backward in time.
The second step is performed using the Piecewise-Parabolic Method (PPM) proposed by \citet{colella:1984}.
As the name suggests, PPM employs piecewise-parabolic functions.
The third and final step is computed easily, as the reconstruction functions consist of parabolas that preserve the local mass.

It is worth noting that the reconstruction function can be constructed using functions other than parabolas.
In fact, PPM can be seen as an
extension of the Piecewise-Linear method proposed by \citet{vanleer:1977}, which,
in turn, was inspired by the Piecewise-Constant method introduced by \citet{godunov:1959}. 
Additionally, other schemes inspired by PPM have been proposed in the literature utilizing
higher-order polynomials, such as quartic polynomials \citep{white:2008}. For a
comprehensive review of general piecewise-polynomial reconstruction, we recommend
referring to the technical report by \citet{engwirda:2016}, \citet{lauritzen:2011}, and the
references therein.

The PPM approach has become popular in the literature for gas dynamics simulations, astrophysical 
phenomena modeling \citep{woodward:1986}, and later on atmospheric simulations \citep{carpenter:1990}. 
Indeed, PPM has been implemented in the FV3 dynamical core on its latitude-longitude grid \citep{lin:2004}
and cubed-sphere \citep{putman:2007} versions.
Although many other shapes for the basis functions and higher-order schemes are available in the literature, 
\citet{harris:2021} points out that the PPM scheme suits the needs of FV3 well. It is a flexible method that
can be modified to ensure low diffusivity or shape preservation, for example.
Additionally, a finite-volume numerical method usually requires monotonicity constraints, which, according 
to Godunov's order barrier theorem \citep{wesseling:2001}, limit the order of convergence to at most 1. 
Therefore, a higher-order scheme needs to strike a well-balanced trade-off between increasing computational 
cost and potential benefits.

This chapter begins with a basic review of one-dimensional advection equation in the integral form
in Section \ref{chp-adv1d-sec1}. In Section \ref{chp-adv1d-sec2-fvsl}, we establish the framework for general
one-dimensional finite-volume Semi-Lagrangian schemes. Section \ref{chp-adv1d-sec-dp} presents
methods for computing the departure point. The PPM reconstruction is described in Section \ref{chp-adv1d-sec-recon},
while Subsection \ref{chp-adv1d-sec-mono} introduces different approaches to ensure the monotonicity of parabolas.
Section \ref{chp-adv1d-sec-flux} focuses on the description and investigation of the PPM flux computation.
Section \ref{chp-adv1d-sec-numerical-exp}
presents numerical results using the PPM scheme for the advection equation.
Finally, Section \ref{chp-adv1d-sec-conclusion} presents some concluding remarks.
The application of PPM to solve two-dimensional problems will be addressed in Chapter \ref{chp-2d-fv}.

\section{One-dimensional advection equation in integral form}
\label{chp-adv1d-sec1}

\subsection{Notation}
\label{chp-adv1d-sec-not}
Before introducing the FV-SL schemes, let us establish some notation by introducing
the concepts of a $\Delta x$-grid, a $\Delta t$-temporal grid, and the
$(\Delta x, \Delta t, \lambda)$-discretization, as well as the concept of grid function/winds.
In this chapter, we will use the notation $\Omega=[a,b]$ to represent the interval under consideration,
and $\nu$ to represent a non-negative integer indicating the number of ghost cell layers in each boundary.
We also use the notations $\mathbb{R}^{N}_{\nu}:=\mathbb{R}^{N+2\nu}$ and
$\mathbb{R}^{N+1}_{\nu}:=\mathbb{R}^{N+1+2\nu}$.
\begin{definition}[$\Delta x$-grid]\label{chp-adv1d-def-dxgrid}
	For a given interval $\Omega$ and a positive real number $\Delta x$ such that 
    $\Delta x = (b-a)/N$ for some positive integer $N$, 
	we say that $\Omega_{\Delta x}= \{X_i \}_{i=-\nu+1}^{N+\nu}$ is a $\Delta x$-grid for $\Omega$ if
	\begin{align*}
        X_i = [x_{i-\frac{1}{2}},x_{i+\frac{1}{2}}] = [a+(i-1)\Delta x, a+i\Delta x],
    \end{align*}
	and $\Delta x = x_{i+\frac{1}{2}} - x_{i-\frac{1}{2}}$. 
	Each $X_i$ is referred to as a control volume or cell, and $x_{i-\frac{1}{2}}$ and 
	$x_{i+\frac{1}{2}}$ are the edges of the control volume $X_i$.
	The cell centroid is defined by
    \begin{align*}
    x_i = \frac{1}{2}(x_{i+\frac{1}{2}} + x_{i-\frac{1}{2}}),\quad \forall i = -\nu+1, \ldots, N+\nu,
    \end{align*}
	and $\Delta x$ is the cell length.
\end{definition}
\begin{remark}
If $1 \leq i \leq N$, we refer to $i$ as an interior index;
otherwise, $i$ is considered a ghost cell index and we say the $X_i$ is a ghost cell.
\end{remark}

\begin{figure}[!htb]
	\centering
	\includegraphics[width=1\linewidth]{1d_grid}
	\caption{Illustration of a $\Delta x$-grid with $N=4$ cells in its interior (in black) 
         and $\nu=2$ ghost cell layers (in gray).
	 The edges are denoted by squares and the cell centroids are denoted using circles.\label{chp-adv1d-sec1-grid1d}}
\end{figure}

\begin{definition}[$\Delta t$-temporal grid]
	For a given interval $[0,T]$ and a positive real number $\Delta t$ such that $\Delta t = T/N_T$
    for some positive integer $N_T$, we say that  $T_{\Delta T}= \{T_n\}_{n=0}^{N_T}$ a $\Delta t$-temporal grid for $[0,T]$ if
    \begin{align*}
	T_n = [t^n, t^{n+1}], \quad t^n = n\Delta t, \quad \Delta t = \frac{T}{N_T}, \quad \forall n = 0, \ldots, N_T.
    \end{align*}
    In this context, we also define $t^{n+\frac{1}{2}} = \frac{t^n+t^{n+1}}{2}$.
\end{definition}

\begin{definition}[$(\Delta x,\Delta t, \lambda)$-discretization]
\label{chp-adv1d-def-dxtimegrid}
	Given $\Omega \times [0,T]$ and positive real numbers $\Delta x$ and $\Delta t$,
    we say that $(\Omega_{\Delta x}, T_{\Delta t})$ is a $(\Delta x, \Delta t, \lambda)$-discretization 
    of $\Omega \times [0,T]$ if $\Omega_{\Delta x}$ is a $\Delta x$-grid for $\Omega$, 
    ${T}_{\Delta t}$ is a $\Delta t$-temporal grid for $[0,T]$, and $\frac{\Delta t}{\Delta x} = \lambda$.
\end{definition}
\begin{remark}
	Whenever we refer to a $\Delta x$-grid, a $\Delta t$-temporal grid, or a $(\Delta x, \Delta t, \lambda)$-discretization, 
	$X_i$, $N$, $t^n$, and $N_T$ are assumed to be implicitly defined.
\end{remark}
Next, we introduce the definitions of grid functions at cell centroids and edges.
\begin{definition}[$\Delta x$-grid function]
	For a $\Delta x$-grid, we say that $Q$ is a $\Delta x$-grid function if
	$Q = (Q_{-\nu+1}, \ldots, Q_{N+\nu}) \in \mathbb{R}^{N}_{\nu}$.
\end{definition}
\begin{definition}[$\Delta x$-grid wind]
	For a $\Delta x$-grid, we say that $u$ is a $\Delta x$-grid wind if
	$u = (u_{-\nu+\frac{1}{2}}, \ldots, u_{N+\nu+\frac{1}{2}}) \in \mathbb{R}^{N+1}_{\nu}$.
\end{definition}
The definition of a $\Delta x$-grid wind is based on the Arakawa grids \citep{arakawa:1977}.
Considering functions $q, u: \Omega \times[0,T] \to \mathbb{R}$ and a $(\Delta x,\Delta t, \lambda)$-discretization
of $\Omega \times[0,T]$, we introduce the grid functions $q^n \in \mathbb{R}^{N}_{\nu}$ and $u^n \in \mathbb{R}^{N+1}_{\nu}$
where ${q}^n_{i} \approx {q}(x_i, t^{n})$ and $u^n_{i+\frac{1}{2}} = u(x_{i+\frac{1}{2}},t^n)$.
The grid function $q^n$ approximate the discrete values of $q$ at cell centroids and $u^n$ represents the values of $u$ approximates edges, both
for each time level $t^n$ (Figure \ref{chp-adv1d-sec1-grid1d-function}).


In this Chapter, our focus lies on periodic grid functions.
We define a $\Delta x$-grid function $Q$ as periodic if it satisfies the following conditions:
\begin{align*}
    Q_{i} &= Q_{N+i}, \quad i=-\nu+1, \ldots, 0,\\
    Q_{i} &= Q_{i-N}, \quad i=N+1, \ldots, N+\nu.
\end{align*}
Similarly, we define a $\Delta x$-grid wind as periodic if it meets the following requirements:
\begin{align*}
    u_{i-\frac{1}{2}} &= u_{N+i+\frac{1}{2}} , \quad i=-\nu, \ldots, -1,\\
    u_{i+\frac{1}{2}} &= u_{i+\frac{1}{2}-N} , \quad i=N+1, \ldots, N+\nu.
\end{align*}
We use the notation $\mathbb{P}^{N}_{\nu}$ and $\mathbb{P}^{N+1}_{\nu}$ to
represent the spaces of periodic $\Delta x$-grid functions and winds, respectively.
\begin{figure}[!htb] 
\centering 
\includegraphics[width=1\linewidth]{1d_grid_function} 
\caption{Illustration of $\Delta x$-grid function $Q$ (black circles)
and a $\Delta x$-grid wind $u$ (blue squares) and its ghost cell
values (in gray) assuming periodicity.\label{chp-adv1d-sec1-grid1d-function}}
\end{figure}

Given $Q \in \mathbb{P}^{N}_{\nu}$, we define the $p$-norm as
\begin{equation}
	\label{chp-adv1d-sec-not1}
	\|Q\|_{p,\Delta x}=
	\begin{cases}
		\bigg( \sum_{i=1}^{N} |Q_i|^p \bigg)^{\frac{1}{p}} & \text{if } 1\leq p < \infty,\\
		\max_{i=1, \ldots, N}{|Q_i|} & \text{otherwise },
	\end{cases}
\end{equation}
which is indeed a norm for periodic grid functions.
Using a similar notation as in \citet{engwirda:2016}, we define the stencil and a grid function evaluated on a stencil as follows.
\begin{definition}[Stencil]
	For a $\Delta x$-grid, and each $i = 0, \ldots, N$, we define a stencil as a set of the form
	$\mathcal{S}_{i+\frac{1}{2}} = \{i-r+1, \ldots, i-1, i, i+1, \ldots, i+s\} \subset\{-\nu+1, \ldots, N+\nu\}$.
\end{definition}
\begin{definition}[Grid function restricted to a stencil]
	For a $\Delta x$-grid, a stencil $\mathcal{S}_{i+\frac{1}{2}}$,
	 and a $\Delta x$-grid function $Q$, we define $Q(\mathcal{S}_{i+\frac{1}{2}}) = (Q_k)_{k \in \mathcal{S}_{i+\frac{1}{2}}}$.
\end{definition}
These definitions provide the necessary notation for describing grid functions and their evaluations on stencils.
To achieve a more compact notation in some situations, we introduce the centered difference notation:
\begin{equation}
    \label{chp-adv1d-sec-adv-eq5}
	\delta_x {g}(x_i,t) = 
	{g}(x_{i+\frac{1}{2}},t) - 
	{g}(x_{i-\frac{1}{2}},t),
\end{equation}
for any function $g: \Omega \times [0,T] \to \mathbb{R}$.
Additionally, we introduce the average value of $q$ in the $i$-th control volume at time $t$, denoted as ${Q}_i(t)$, defined by:
\begin{equation}
	\label{chp-adv1d-sec1-not2}
	{Q}_i(t) = \frac{1}{\Delta x} \int_{x_{i-\frac{1}{2}}}^{x_{i+\frac{1}{2}}} {q}(x,t) \,dx.
\end{equation}
Moreover, we define the $\Delta x$-grid function of average values as $Q(t) = (Q_i(t))_{i=-\nu+1}^{N+\nu}$.
Here, $Q_i(t)$ represents the average value of $q$ in the $i$-th control volume at time $t$.

For the consideration of periodic boundary conditions, we can define spaces of periodic functions over 
the interval $\Omega$ as follows:
\begin{align*}
	\mathcal{S}_P(\Omega) &= \{q:\mathbb{R}\times[0,+\infty[\to \mathbb{R}: q(x+b-a,t)=q(x,t), \quad \forall x \in \mathbb{R}, \quad t\geq0\}.
\end{align*}
Similarly, the space of $k$-times periodically differentiable functions $\mathcal{C}_P^k(\Omega)$ can be defined as:
\begin{align*}
	\mathcal{C}_P^k(\Omega) &= \mathcal{S}_P(\Omega)\cap \mathcal{C}^k(\mathbb{R}\times[0,\infty[),
\end{align*}
where $\mathcal{C}^k(\mathbb{R}\times[0,+\infty[)$ denotes the space of functions that are $k$ 
times continuously differentiable in both the spatial and temporal variables.
In summary, $\mathcal{S}_P(\Omega)$ represents the space of periodic functions, and $\mathcal{C}_P^k(\Omega)$
represents the space of $k$-times periodically differentiable functions over the interval $\Omega$ subject to periodic boundary conditions.

\subsection{The 1D advection equation}
In this section, we will derive the integral form of the 1D advection equation with periodic boundary conditions over the interval $\Omega$.
What is going to be presented here follows \citet{leveque:1990,leveque:2002} closely.
The advection equation with periodic boundary conditions in its differential form is given by:
\begin{equation}
	\label{chp-adv1d-sec-adv-eq1}
	\begin{cases}
		[{\partial_t q} + {\partial_x (uq)}](x, t)
		= 0, \quad \forall (x,t) \in \mathbb{R}\times ]0, +\infty[,\\
		{q}(a, t) = {q}(b, t), \quad \forall t\geq 0,\\
		q_0(x) = q(x,0), \quad \forall x \in \Omega.
	\end{cases}
\end{equation}
Here, $q \in \mathcal{C}_P^1{(\Omega)}$ represents the advected quantity, and $u \in \mathcal{C}_P^1{(\Omega)}$ represents the velocity.
We will focus on Equation \eqref{chp-adv1d-sec-adv-eq1} over the domain $D = \Omega\times[0,T]$, where $T>0$ is a finite time.
A strong or classical solution to the advection equation is defined as a function ${q}\in\mathcal{C}^1_P(\Omega)$ 
and satisfies Equation \eqref{chp-adv1d-sec-adv-eq1}.
In order to deduce the integral form of Equation \eqref{chp-adv1d-sec-adv-eq1}, we consider
$[x_1, x_2]\times[t_1,t_2]\subset D$. 
Integrating Equation \eqref{chp-adv1d-sec-adv-eq2} over $[x_1, x_2]$, we obtain:
\begin{equation}
    \label{chp-adv1d-sec-adv-eq2}
	\frac{d}{dt} \int_{x_1}^{x_2} q(x,t) \,dx =  
	-({(uq)}(x_2,t) - {(uq)}(x_1,t)) ,
\end{equation}
and integrating Equation \eqref{chp-adv1d-sec-adv-eq2} over $[t_1,t_2]$, we get
\begin{equation}
    \label{chp-adv1d-sec-adv-eq3}
    \int_{x_1}^{x_2} q(x,t_2) \,dx=  \int_{x_1}^{x_2} q(x,t_1)
	-\bigg( \int_{t_1}^{t_2} 
	{(uq)}(x_2, t) \,dt - 
	\int_{t_1}^{t_2}{(uq)}(x_1, t) \,dt \bigg).
\end{equation}
The presented problem, Problem \ref{chp-adv1d-sec2-prob1}, aims to find a solution, called weak solution, to the advection equation
in its integral form, considering the given initial condition ${q}_0$ and velocity function $u$.

\theoremstyle{plain}
\newtheorem{prob}{Problem}[chapter]
\begin{prob}
	\label{chp-adv1d-sec2-prob1}
	Given an initial condition ${q}_0$ and a velocity function $u$  we would like to find a weak solution ${q}$
	of the advection equation in the integral form:
	\begin{equation*}
	        \int_{x_1}^{x_2} {q}(x, t_2) \,dx =
       		\int_{x_1}^{x_2} {q}(x, t_1) \,dx +
        	\int_{t_1}^{t_2} {(uq)}(x_1, t) \,dt -
		\int_{t_1}^{t_2}{(uq)}(x_2, t) \,dt ,
	\end{equation*}
	$\forall [x_1, x_2]\times[t_1, t_2] \subset \Omega \times [0,T]$,
	and
	${q}(x,0) = {q}_0(x)$, $\forall x \in \Omega$, $q(a,t)=q(b,t)$, $\forall t \in [0,T]$.
\end{prob}
We point out that, for Problem \ref{chp-adv1d-sec2-prob1}, the total mass in $\Omega$ at time $t$ defined by:
\begin{equation*}
{M}_{[a,b]}(t) = \int_{a}^{b} q(x,t) \,dx,
\end{equation*}
remains constant over time, i.e.,
\begin{equation*}
	{M}_{[a,b]}(t) = {M}_{[a,b]}(0), \quad \forall t \in [0,T].
\end{equation*}
This conservation of total mass property is highly desirable for numerical schemes aiming
to approximate general conservation law solutions accurately.

Applying the steps from Equation \eqref{chp-adv1d-sec-adv-eq1} to Equation \eqref{chp-adv1d-sec-adv-eq3} in reverse order,
one can verify that if ${q}$ is a weak solution and $q \in \mathcal{C}^1_P{(\Omega)}$, then it satisfies Equation \eqref{chp-adv1d-sec-adv-eq1}.
Therefore, Equation \eqref{chp-adv1d-sec-adv-eq1} and Problem \eqref{chp-adv1d-sec2-prob1} are equivalent when $q \in \mathcal{C}^1_P{(\Omega)}$.
However, Problem \eqref{chp-adv1d-sec2-prob1} can be formulated for functions that are not $\mathcal{C}^1$ and have discontinuities.
In fact, Problem \eqref{chp-adv1d-sec2-prob1} only requires that $q$ and $uq$ are locally integrable.

It is worth noting that Equation \eqref{chp-adv1d-sec-adv-eq3} holds for all $x_1, x_2, t_1$, and $t_2$ such that $[x_1, x_2]
\times [t_1, t_2] \subset D$. Therefore, let us consider a $(\Delta x, \Delta t, \lambda)$-discretization of $D$ and 
rewrite Equation \eqref{chp-adv1d-sec-adv-eq3} in terms of this discretization. By replacing $t_1, t_2, x_1$, and $x_2$ with
$t^{n}, t^{n+1}, x_{i-\frac{1}{2}}$, and $x_{i+\frac{1}{2}}$, respectively, in Equation \eqref{chp-adv1d-sec-adv-eq3}, we obtain:
\begin{equation}
    \label{chp-adv1d-sec-adv-eq4}
	\begin{aligned}
		{Q}_i(t^{n+1}) =  {Q}_i(t^{n}) -
		\frac{1}{\Delta x}\bigg( \int_{t^{n}}^{t^{n+1}}
        	{(uq)}(x_{i+\frac{1}{2}}, t) \,dt -
		\int_{t^{n}}^{t^{n+1}}{(uq)}(x_{i-\frac{1}{2}}, t) \,dt \bigg), \\
		\quad \forall i = 1, \ldots, N,
		\quad \forall n = 0, \ldots, N_T-1.
	\end{aligned}
\end{equation}
To achieve a more compact notation, we use the centered difference notation
and then Equation \eqref{chp-adv1d-sec-adv-eq4} can be rewritten as:
\begin{equation}
    \label{chp-adv1d-sec-adv-eq6}
    {Q}_i(t^{n+1}) =  {Q}_i(t^{n}) -
	\frac{1}{\Delta x} \delta _x\bigg( \int_{t^{n}}^{t^{n+1}}
        {(uq)}(x_{i}, t) \,dt \bigg),
        \quad \forall i = 1, \ldots, N,
        \quad \forall n = 0, \ldots, N_T-1.
\end{equation}

Now we can define a discretized version of Problem \ref{chp-adv1d-sec2-prob1} as Problem \ref{chp-adv1d-sec2-prob2}.
\begin{prob}
	\label{chp-adv1d-sec2-prob2}
	Let us consider the framework of Problem \ref{chp-adv1d-sec2-prob1} and a $(\Delta x, \Delta t, \lambda)$-discretization of 
	$\Omega \times [0,T]$. Since we are operating within the framework of Problem \ref{chp-adv1d-sec2-prob1}, the following relationship holds:
	\begin{equation}
		\label{1d-fvexact-scheme}
		{Q}_i(t^{n+1}) = {Q}_i(t^{n}) - \lambda \delta_x\bigg( \frac{1}{\Delta t}\int_{t^{n}}^{t^{n+1}}{(uq)}(x_{i}, t) \,dt \bigg),
		 \quad \forall i = 1, \ldots, N, \quad \forall n = 0, \ldots, N_T-1,
	\end{equation}
	where ${Q}_i(t) = \frac{1}{\Delta x}\int_{x_{i-\frac{1}{2}}}^{x_{i+\frac{1}{2}}} {q}(x,t) \,dx$. 
	Our objective now is to determine the values ${Q}_i(t^{n})$, $\forall i = 1, \ldots, N$, $\forall n = 0, \ldots, N_T-1$,
	given the initial values ${Q}_i(0)$, $\forall i = 1, \ldots N$. 
	In other words, we aim to find the average values of ${q}$ in each control volume $X_i$ at the specified time instances.
\end{prob}
It is important to note that no approximations have been made in problems \eqref{chp-adv1d-sec2-prob1} and \eqref{chp-adv1d-sec2-prob2}. 
In Equation \eqref{1d-fvexact-scheme}, we divided and multiplied by $\Delta t$ to interpret
$\frac{1}{\Delta t}\int_{t^{n}}^{t^{n+1}}({uq})(x_{i\pm \frac{1}{2}}, t) \,dt$ 
as a time-averaged flux. This interpretation is useful for deriving finite-volume schemes.

In Problem \ref{chp-adv1d-sec2-prob2}, we need to approximate the time-averaged flux at the cell edges $x_{i\pm\frac{1}{2}}$
to derive a finite-volume scheme. This flux, in principle, requires knowledge of $q$ over the entire interval $[t^n, t^{n+1}]$. 
To overcome this, we can express the temporal integral as a spatial integral at time $t^n$. 
This approach avoids the need for information about $q$ throughout the entire interval $[t^n, t^{n+1}]$. 
Furthermore, this spatial integral domain is closely related to the definition of the 
departure point. 

To introduce the definition of departure point, for each $s \in [t^n,t^{n+1}]$,
we consider the following Cauchy problem backward in time:
\begin{equation}
	\label{chp-sec-flux:analysis-eq3}
	\begin{cases}
		\partial_t x^d_{i+\frac{1}{2}} (t,s) = u(x^d_{i+\frac{1}{2}}(t,s) ,t),\quad t\in[t^{n},s] \\
		x^d_{i+\frac{1}{2}}(s,s) = x_{i+\frac{1}{2}}.
	\end{cases}
\end{equation}
The point $x^d_{i+\frac{1}{2}}(t^n,s)$ is called departure point at time $t^n$
of the point $x_{i+\frac{1}{2}}$ at time $s$.
In Figure \ref{chp-adv1d-sec1-grid1d-dep} we illustrate the departure point idea.

\begin{figure}[!htb]
	\centering
	\includegraphics[width=1\linewidth]{1d_grid_departure}
	\caption{Illustration of the departure point of the cell edges from time $t^{n+1}$ to $t^n$.\label{chp-adv1d-sec1-grid1d-dep}}
\end{figure}
Integrating Equation $\eqref{chp-sec-flux:analysis-eq3}$ over the interval
$[t,s]$, we get:
\begin{equation}
	\label{chp-sec-flux:analysis-eq4}
	x^d_{i+\frac{1}{2}} (t,s) = x_{i+\frac{1}{2}} - \int_{t}^{s}u(x^d_{i+\frac{1}{2}}(\theta,s),\theta)  \,d\theta.
\end{equation}
In the following Proposition, we show how the time-averaged flux is 
related to a spatial integral over a interval depending on departure points.
\begin{prop}
	\label{chp-adv1d-sec-flux:prop1}
	Assume the framework of Problem \ref{chp-adv1d-sec2-prob2}.
	If $q$ and $u$ are $\mathcal{C}^1$ functions, then:
	\begin{align}
		\label{chp-adv1d-sec-flux:approx1}
		\int_{t^n}^{t^{n+1}} (uq)(x_{i+\frac{1}{2}},s) \,ds = 
		\int^{x_{i+\frac{1}{2}}}_{x^d_{i+\frac{1}{2}}(t^n,t^{n+1})} q(x,t^n)\,dx
	\end{align}
\end{prop}
\begin{proof}
Using the Leibniz rule for integration (Theorem \ref{anexo-numint-lr} with 
$f(s,\theta) = u\big(x^d_{i+\frac{1}{2}}(\theta,s)\big)$),
in Equation \eqref{chp-sec-flux:analysis-eq4}, it follows that:
	\begin{align}
		\begin{split}
			\label{dxds}
			{\partial_s x_{i+\frac{1}{2}}^d} (t,s) &= - \bigg({u}(x_{i+\frac{1}{2}},s) + 
			\int_{t}^{s} \frac{d{u}}{ds}\big( x_{i+\frac{1}{2}}^d(\theta,s),\theta\big) \,d\theta \bigg)\\
			&=- {u}(x_{i+\frac{1}{2}},s) -
			\int_{t}^{s} {\partial_s}{{u}}\big( x_{i+\frac{1}{2}}^d(\theta, s),\theta\big) 
			{\partial_s  x_{i+\frac{1}{2}}^d}(\theta, s)\,d\theta.
		\end{split}
	\end{align}
	Taking the derivative with respect to $t$ of Equation \eqref{dxds}, we have:
	\begin{equation}
			\label{dxds_2}
			{\partial_t }{\partial_s  x_{i+\frac{1}{2}}^d}
			(t,s) = {\partial_x}{{u}}\big(x_{i+\frac{1}{2}}^d(t, s), t\big) 
			{\partial _s} x_{i+\frac{1}{2}}^d (t, s).
	\end{equation}
	Using standard ordinary differential equations techniques (ODE), 
	we get that $ x_{i+\frac{1}{2}}^d$ that solves Equations \eqref{dxds} and \eqref{dxds_2}
	is given by:
	\begin{equation}
			\label{xs_int}
			{\partial_s  x_{i+\frac{1}{2}}^d}(t,s) = -
			\exp{\bigg(\int_{t}^{s} {\partial_x}{{u}}\big( x_{i+\frac{1}{2}}^d(\theta,s),\theta\big)  \,d\theta \bigg)}
			{u}(x_{i+\frac{1}{2}},s).
	\end{equation}
	Computing $q$ on the trajectory give by $x_{i+\frac{1}{2}}^d(t,s)$ and taking
	its time derivative, we obtain:
	\begin{align}
		\label{dqdt}
		\begin{split}
			\frac{d}{dt}q\big( x_{i+\frac{1}{2}}^d(t,s),t\big) &= 
			{\partial_t}q\big( x_{i+\frac{1}{2}}^d(t,s),t\big)+
			({u}{\partial_x } 
			q)\big(x_{i+\frac{1}{2}}^d(t,s),t\big) \\
			&= -{\partial_x}{{u}}\big( x_{i+\frac{1}{2}}^d(t,s),t\big) q \big(x_{i+\frac{1}{2}}^d(t,s),t\big),
		\end{split}
	\end{align}
	where we used that $q$ satisfies the linear advection equation on its differential \eqref{chp-adv1d-sec-adv-eq1} form
   and that $x_{i+\frac{1}{2}}^d(t,s)$ solves Equation \eqref{chp-sec-flux:analysis-eq3}.
	Using again standard ODE techniques, we get that $q$ that solves Equation \eqref{dqdt}
	is given by:
	\begin{equation}
			\label{q_int}
			 q\big( x_{i+\frac{1}{2}}^d(t,s),t\big) = 
			\exp{\bigg(-\int_{t}^{s} {\partial_x}{{u}}\big( x_{i+\frac{1}{2}}^d(\theta,s),\theta\big)  \,d\theta \bigg)}
			 q(x_{i+\frac{1}{2}},s).
	\end{equation}
	Notice that if ${u}$ does not depend on $x$, then $q$ is constant along the trajectory $ x_{i+\frac{1}{2}}^d(t,s)$.
	
	Let us consider the mapping $s\in[t^n,t^{n+1}] \to  x_{i+\frac{1}{2}}^d(t^n,s)$. 
	Integrating $q$ over all departure points at time $t^n$ from $x_{i+\frac{1}{2}}$ at time $s$, we have
	\begin{equation}
		\label{depint_1}
		\int^{ x_{i+\frac{1}{2}}^d(t^n,t^{n+1})}_{ x_{i+\frac{1}{2}}^d(t^n,t^{n}) = x_{i+\frac{1}{2}}}   
		q(x,t^n)\,dx 
		= \int_{t^n}^{t^{n+1}}  q\big( x_{i+\frac{1}{2}}^d(t^n,s),t^n\big){\partial_s}{ x_{i+\frac{1}{2}}^d} (t^n,s)\,ds,
	\end{equation}
	where we are just using the variable change integration formula.
	Then, it follows from Equations  \eqref{xs_int}
	and \eqref{q_int} with $t=t^n$ that:
	\begin{equation}
		\label{depint_2}
			\int^{ x_{i+\frac{1}{2}}^d(t^n,t^{n+1})}_{x_{i+\frac{1}{2}}} q(x,t^n)\,dx 
			= -\int_{t^n}^{t^{n+1}} ({u} q)(x_{i+\frac{1}{2}},s) \,ds, 
	\end{equation}
	which is the desired formula.
\end{proof}
With the aid of Proposition \ref{chp-adv1d-sec-flux:prop1}, we can rewrite Problem \ref{chp-adv1d-sec2-prob2}
in terms of the departure point, avoiding the need for knowledge about $q$ over the
entire interval $[t^n, t^{n+1}]$. This is described in Problem \ref{chp-adv1d-sec2-prob3}:
\begin{prob}
    \label{chp-adv1d-sec2-prob3}
	Assume the framework of Problem \ref{chp-adv1d-sec2-prob1}
    and a $(\Delta x, \Delta t, \lambda)$-discretization of $\Omega \times [0,T]$.
	Since we are in the framework of Problem \ref{chp-adv1d-sec2-prob1}, it follows that:
\begin{equation}
	\label{1d-fvslexact-scheme}
	\begin{split}
{Q}_i(t^{n+1}) =  {Q}_i(t^{n}) -
\lambda
\bigg( \frac{1}{\Delta t}\int_{X(t^n,t^{n+1};x_{i+\frac{1}{2}})}^{x_{i+\frac{1}{2}}}{q}(x, t^n) \,dx-
\frac{1}{\Delta t} \int_{x_{i-\frac{1}{2}}^d(t^n,t^{n+1})}^{x_{i-\frac{1}{2}}}{q}(x, t^n) \,dx \bigg),\\
\quad \forall i = 1, \ldots, N,
\quad \forall n = 0, \ldots, N_T-1,
	\end{split}
\end{equation}
	where ${Q}_i(t) = \frac{1}{\Delta x}
	\int_{x_{i-\frac{1}{2}}}^{x_{i+\frac{1}{2}}} {q}(x,t) \,dx$.
	Our problem now consists of finding the values ${Q}_i(t^{n})$, 
	$\forall i = 1, \ldots, N$, $\forall n = 0, \ldots, N_T-1$,
	given the initial values ${Q}_i(0)$, $\forall i = 1, \ldots N$.
	In other words, we would like to find the average values of ${q}$
	in each control volume $X_i$ at the considered time instants.
\end{prob}
At each time step $t^n$, we compute the values of ${Q}_i(t^{n+1})$ based on ${Q}_i(t^{n})$ and the integrals 
of $q(x, t^n)$ over specific intervals. These intervals are defined by the departure points 
$X(t^n, t^{n+1}; x_{i+\frac{1}{2}})$ and $X(t^n, t^{n+1}; x_{i-\frac{1}{2}})$.
To perform the computations, we need to determine the departure points from the edges of all control volumes
and calculate the required integrals. This idea serves as the motivation for defining finite-volume 
Semi-Lagrangian schemes. These schemes involve estimating the departure points and reconstructing the 
function $q$ at time $t^n$ using its average values $Q_i(t^n)$, which enables us to compute the necessary integrals.

\section{The finite-volume Semi-Lagrangian approach}
\label{chp-adv1d-sec2-fvsl}
Finally, we define the 1D FV-SL scheme problem as follows in Problem \ref{chp-adv1d-sec2-prob3}.
\begin{prob}[1D FV-SL scheme]
	\label{chp-adv1d-sec2-prob4}
	Assume the framework defined in Problem \ref{chp-adv1d-sec2-prob3}.
	The finite-volume Semi-Lagrangian approach of Problem \ref{chp-adv1d-sec2-prob3}
	consists of finding a scheme of the form:
	\begin{equation}
		\label{1d-fv-scheme}
		{Q}_{i}^{n+1} = {Q}_{i}^{n} -
		\lambda({F}_{i+\frac{1}{2}}^{n} - {F}_{i-\frac{1}{2}}^{n}),
		\quad \forall i = 1, \ldots, N,
		\quad \forall n = 0, \ldots, N_T-1,
	\end{equation}
	where ${Q}^{n} \in \mathbb{P}^{N}_{\nu}$ is intended to be an approximation
	of ${Q}(t^{n})\in \mathbb{P}^{N}_{\nu}$ in some sense. We define
	${Q}_{i}^{0} = {Q}_i(0)$ or ${Q}_{i}^{0} = {q}^{0}_{i}$.
	The terms ${F}_{i\pm\frac{1}{2}}^{n}$ are known as numerical flux and are given by
	\begin{equation}
		{F}_{i\pm\frac{1}{2}}^{n} =
		\frac{1}{\Delta t}
		\int_{\tilde{x}_{i\pm\frac{1}{2}}^n}^{x_{i\pm\frac{1}{2}}}\tilde{q}(x; Q^n) \,dx,
	\end{equation}
	where ${\tilde{x}_{i\pm\frac{1}{2}}}^n$ is an estimate of the departure point $x_{i-\frac{1}{2}}^d(t^n,t^{n+1})$,
	and $\tilde{q}$ is a reconstruction function for $q$ built with the values $Q^n$.
	Thus, ${F}_{i\pm\frac{1}{2}}^{n}$ approximates
	$\frac{1}{\Delta t} \int_{x_{i\pm\frac{1}{2}}^d(t^n,t^{n+1})}^{x_{i\pm \frac{1}{2}}}{q}(x, t^n) \,dx$.
\end{prob}
For a 1D FV-SL the discrete total mass at the time-step $n$ is given by
\begin{equation}
	\label{1d-fv-mass}
	M^n =  \Delta x \sum_{i=1}^N Q_i^n.
\end{equation}
Therefore, the discrete total mass is constant for a 1D-FV scheme,
which follows from a straightforward computation:
\begin{align*}
	M^{n+1} =  \Delta x \sum_{i=1}^N Q_i^{n+1}
					= M^{n} - \Delta t  \sum_{i=1}^N (F^n_{i+\frac{1}{2}}- F^n_{i-\frac{1}{2}})
					= M^{n} - \Delta t (F^n_{N+\frac{1}{2}}- F^n_{\frac{1}{2}})
					= M^{n},
\end{align*}
where we are using that $F^n_{N+\frac{1}{2}} = F^n_{\frac{1}{2}}$, since we are assuming periodic boundary
conditions.

We would like to highlight an important relationship between the average values of $q$ and
its values at the cell centroids. In Problem \ref{chp-adv1d-sec2-prob4}, we mentioned that 
the initial condition can be represented as $q_i^0$ instead of $Q_i(0)$.
Moreover, when analyzing the convergence of a FV-SL scheme, it is useful
to compare $Q_i^n$ with $q_i^n$ since computing $Q_i(t^n)$ requires evaluating an analytical
integral, which can be challenging in certain cases. In Proposition \ref{prop-bound-centroid},
we provide a simple proof that $q_i^n$ approximates $Q_i(t^n)$ with second-order error
when $q$ is twice continuously differentiable.
\begin{prop}
	\label{prop-bound-centroid}
	If $q \in \mathcal{C}^2_P(\Omega)$, then $Q_i(t^n)-q_i^n = C_1 \Delta x^2$, where 
	$C_1 = \frac{1}{24}\frac{\partial^2 q}{\partial x^2} (\eta, t^n)$,  $\eta \in X_i$.
\end{prop}
\begin{proof}
	Just apply Theorem \ref{prop-bound-midpoint1d} for the function $q(x,t^n)$.	
\end{proof}
Hence, 1D FV-SL schemes may be conceptualized as schemes that update the centroid values.
The Problem of the convergence of 1D FV-SL schemes is addressed in Section \ref{convergence-1dfvsl}.
Now we are going to address the problem of the departure point estimation and the reconstruction problem.
