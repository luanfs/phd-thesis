%!TeX root=../tese.tex
%("dica" para o editor de texto: este arquivo é parte de um documento maior)
% para saber mais: https://tex.stackexchange.com/q/78101

%% ------------------------------------------------------------------------- %%

% "\chapter" cria um capítulo com número e o coloca no sumário; "\chapter*"
% cria um capítulo sem número e não o coloca no sumário. A introdução não
% deve ser numerada, mas deve aparecer no sumário. Por conta disso, este
% modelo define o comando "\unnumberedchapter".
\chapter{Introduction}
\label{cap:introduction}

\enlargethispage{.5\baselineskip}

\section{Background}

Weather and climate predictions are recognized as a good for mankind,
due to the information they yield for diverse activities. 
For instance, short-range forecasts are useful for public use, while
medium-range forecasts are helpful for industrial activities and agriculture. 
Seasonal forecasts (one up to three months) are important to energy planning and agriculture.
At last, longer-range forecasts (one century, for instance) are useful for climate change 
projections that are important for government planning.

The first global Numerical Weather Prediction models emerged in the 1960s
with applications to weather, seasonal and climate forecasts. 
All these applications are essentially based on the same set of Partial Differential Equations
(PDEs) but with distinct time scales \citep{stan:2008}. These PDEs are defined on the sphere
and model the evolution of the atmospheric fluid given the initial conditions.
One important component of global models is the dynamical core, which is responsible
for solving the PDEs that governs the atmosphere dynamics on grid-scale \citep{will:2007}. 
The development of numerical methods for dynamical cores has been an active research area since the 1960s.

Global models use the sphere as the computational domain and therefore they require a discretization
of the sphere. 
The first global models used the latitude-longitude grid (Figure \ref{latlon-grid}), which is
very suitable for finite-differences schemes due to its orthogonality \citep{will:2007}. The major drawback of the latitude-longitude grid
is the clustering of points at the poles, known as the ``pole problem'', which leads to extremely
small time steps for explicit-in-time schemes due to the Courant-Friedrichs-Lewy (CFL) condition, 
making these schemes computationally very expensive \citep{randall:2020}.

The most successful method adopted in global atmospheric dynamical cores that overcomes the CFL 
restriction is the Semi-Implicit Semi-Lagrangian (SI-SL) scheme \citep{randall:2018}, which 
emerged in the 1980s and consists of the Lagrangian advection scheme applied at each time-step 
and the solution of fast gravity waves implicitly, allowing very large time steps despite the pole problem.
The SI-SL approach combined with finite differences is still used nowadays, for instance in 
the UK Met Office global model ENDGame \citep{wood:2014}.
The expensive part of the  SI-SL approach is to solve an elliptic equation at each time step,
that comes from the semi-implicit discretization, which requires global data communication,
being inefficient to run in massive parallel supercomputers. Besides that, 
Semi-Lagrangian schemes are inherently non-conservatives for mass,
which is critical for climate forecasts \citep{will:2007}.

\begin{figure}
	\centering
	\begin{subfigure}{.3\linewidth}
		\centering
		\includegraphics[width = \linewidth]{latlon_sphere}
		\caption{Latitude-longitude grid}\label{latlon-grid}
	\end{subfigure}%
  \hspace{1em}% Space between image A and B
\begin{subfigure}{.3\linewidth}
	\centering
	\includegraphics[width = \linewidth]{gnomonic_equiangular_15_sphere}
	\caption{Cubed-sphere}\label{cs-grid}
\end{subfigure}%
	\hspace{2em}% Space between image B and C
	\begin{subfigure}{.3\linewidth}
		\centering
		\includegraphics[width = \linewidth]{icos_pol_nopt_3_edsphere}
		\caption{Icosahedral grid}\label{icos-grid}
	\end{subfigure}
	\begin{subfigure}{.3\linewidth}
		\centering
		\includegraphics[width = \linewidth]{icos_pol_nopt_3_edhxsphere}
		\caption{Pentagonal/Hexagonal grid}\label{voronoi-grid}
	\end{subfigure}
	\begin{subfigure}{.3\linewidth}
		\centering
		\includegraphics[width = \linewidth]{octg_pol_nopt_4_edsphere}
		\caption{Octahedral grid}\label{octg-grid}
	\end{subfigure}
	\caption{Examples of spherical grids: latitude-longitude grid (a) and grids based on Platonic solids (b)-(d).}
\end{figure}

The emergence of the Fast Fourier Transform (FFT) in the 1960s with the work from 
\citet{cooley:1965} allowed the computation of discrete Fourier transforms with
$N\log(N)$ complexity. 
The viability of the usage of FFTs for solving atmospheric flows was shown by \citet{orszag:1970},
using the barotropic vorticity equation on the sphere, and by \citet{eliasen:1970}, using
the primitive equations.
The spectral transform method expresses latitude-longitude grid values, that represent some scalar field,
using truncated spherical harmonics expansions, which consists of Fourier expansions 
in latitude circles and Legendre functions expansions in longitude circles. 
The coefficients in the spectral expansions are known
as spectral coefficients and are usually thought to live in the so-called spectral space.
Given the grid values, the spectral coefficients are obtained by performing a FFT followed by a 
Legendre Transform (LT). 
Conversely, given the spectral coefficients,
the grid values are obtained by performing an inverse LT followed by an inverse FFT.
The main idea of the spectral method is to apply the spectral transform, in order to 
go the spectral space, and evaluate spatial derivatives in the spectral space, which
consists of multiplying the spectral coefficients by constants. 
Then, the method performs the inverse spectral transform
in order to get back to grid space, and the nonlinear terms are treated on the grid space
\citep{krishnamurti:2006}.

The spectral transform makes the use of SI-SL methods computationally cheap, 
since the solution to elliptic problems becomes easy, once the spherical harmonics are
eigenfunctions of the Laplacian operator on the sphere.
Therefore, the spectral transform method gets faster when combined with the SI-SL approach
due to the larger times-steps allowed in this case.
Due to these enhancements, the spectral transform dominated global atmospheric modeling 
\citep{randall:2018} since the 1980s.
Indeed, the spectral method is still used in many current operational Weather Forecasting models such as
the Integrated Forecast System (IFS) from  European Centre for Medium-Range Weather Forecasts (ECMWF),
Global Forecast System (GFS) from National Centers for Environmental Prediction (NCEP) and the
Brazilian Global Atmospheric Model (BAM) \citep{figueroa:2016} from  Center for Weather Forecasting 
and Climate Research [Centro de Previsão de Tempo e Estudos Climáticos (CPTEC)].

With the beginning of the multicore era in the 1990s, the global atmospheric models
started to move towards parallel efficiency aiming to run at very high resolutions. 
Even though the spectral transform expansions have a global data dependency, some parallelization
is feasible among all the computations of FFTs, LTs and their inverses \citep{barros:1995}.
However, the parallelization of the spectral method requires data transpositions in order
to compute FFTs and LTs in parallel.  
These transpositions demand a lot of global communication using, for instance,
the Message Passing Interface (MPI) \citep{zheng:2018}. 
Indeed, the spectral transform becomes the most expensive component of global spectral models
when the resolution is increased due to the amount of MPI communications \citep{mueller:2019}.

The adiabatic and frictionless continuous equations that govern the atmospheric flow have 
conserved quantities. Among them, some of the most important are mass, total energy, 
angular momentum and potential vorticity \citep{thuburn:2011}.
Numerical schemes that are known for having discrete analogous of these conservative properties
are known as mimetic schemes.
As we pointed out, Semi-Lagrangian schemes lack mass conservation. Nevertheless, 
these schemes have been employed in dynamical cores for better computational performance.
However, dynamical cores should have discrete analogous of the 
continuous conserved quantities, especially concerning for longer simulation runs.

Aiming for better performance in massively parallel computers and conservation properties, 
new dynamical cores have been developed since the beginning of the 2000s.
Novel spherical grids have been proposed, in order to avoid the pole problem.
A popular choice are grids based on Platonic solids \citep{stan:2012}.
The construction of these grids relies on a Platonic circumscribed on the sphere and 
the projection of its faces onto the sphere, which leads to quasi-uniform and more isotropic
spherical grids.
Some examples of spherical grids based on Platonic solids employed in the new generation
of dynamical cores are the cubed-sphere (Figure \ref{cs-grid}), icosahedral grid (Figure \ref{icos-grid}), 
the pentagonal/hexagonal or Voronoi grid (Figure \ref{voronoi-grid}) and octahedral grid (Figure \ref{octg-grid}),
which are based on the cube, icosahedron, dodecahedron and octahedron, respectively \citep{ullrich:2017}.

\section{Motivations and the FV3 model}
The cubed-sphere became a popular quasi-uniform grid for the new generation of dynamical cores.
It was originally proposed by \citet{sadourny:1972} and it was revisited by \citet{ronchi:1996}.
Some of the cubed-sphere advantages are: uniformity; quadrilateral structure, 
making the grid indexing trivial; no overlappings; it is cheap to generate.
However, the major drawbacks of the cubed-sphere are: non-orthogonal coordinate system, which
leads to metric terms on the differential operators; discontinuity of the coordinate system
at the cube edges, which may generate numerical noise and
demands special treatment of discrete operators at the cube edges.

Despite of its drawbacks, the cubed-sphere has been adopted in some of the new generation
dynamical cores.
For instance, the cubed-sphere is used in the 
Community Atmosphere Model (CAM-SE) from the NCAR using spectral elements \citep{dennis:2012} and in the
Nonhydrostatic Unified Model of the Atmosphere (NUMA) from the US Navy using Discontinuous Galerkin 
methods \citep{giraldo:2013}. The cubed-sphere was also chosen to be used in the next UK Met Office
global model using mixed finite elements \citep{melvin:2022}.
At last, the Finite­ Volume Cubed-Sphere dynamical core (FV3) from the Geophysical Fluid 
Dynamics Laboratory (GFDL) and the National Oceanic and Atmospheric Administration (NOAA)
\citep{putman:2007,harris:2013} is another example
of new generation dynamical core based on the cubed-sphere.

The FV3 dynamical core is an extension of the Finite-Volume dynamical core (FVcore)
from latitude-longitude grids to the cubed-sphere.
The numerical methods from FVcore started to be developed with the transport scheme from the work \citet{lin:1994},
which is based on the piecewise linear scheme from \citet{vanleer:1977}. 
This scheme was later improved, using the Piecewise Parabolic Method (PPM) \citep{colella:1984, carpenter:1990}
using dimension splitting techniques that guarantee monotonicity and mass conservation,
for the transport equation \citep{lin:1996} and the shallow-water equations \citep{lin:1997}. 
An important feature is that the FVcore combines the  Arakawa C- and D-grids \citep{arakawa:1977},
where the C-grid values are computed in and intermediate time step. 
The full global model was then presented by \citet{lin:2004}.

A computational disadvantage of the FVcore is its Semi-Lagrangian formulation that creates more demand
for MPI communication when computing the accumulated fluxes \citep{lin:1996}.
The FVcore was then adapted to the cubed-sphere grid \citep{putmanthesis:2007, putman:2007}, 
to reach better performance in parallel computers, leading to the FV3 model.
Later, the FV3 also was improved to allow locally refinement grids 
through grid-nesting or grid-stretching \citep{harris:2013}.

Currently, the FV3 model is capable of perfoming hydrostatic and non-hydrostatic atmospheric simulations 
and it was chosen as the new US global weather prediction model, indeed, it replaced the spectral transform
Global Forecast System (GFS) in June, 2019
(\url{https://www.noaa.gov/media-release/noaa-upgrades-us-global-weather-forecast-model}, last access: 19th of March, 2024).
Additionally, the FV3 dynamical core is employed in the GEOS Chem model \citep{martin:2022} from Harvard University, 
in NASA’s next-generation Mars Climate Model \citep{wilson:2022}, and also in the SHiELD model from GFDL \citep{harris:2020}.

However, a well-known problem that occurs on cubed-sphere models that use low-order numerical methods
is the grid imprinting visible due to the coordinate system discontinuity, 
especially at larger scales, leading to the emergence of a wavenumber 4 pattern.
This was reported in the paper of \citet{rancic:2017}, where the authors employ a 
finite-difference numerical scheme on the Uniform Jacobian cubed-sphere using a Arakawa B-grid.
The unpublished report from \citet{whitaker:2015} shows grid imprinting in other models, including 
the FV3.
More recently, \citet{mouallem:2023} has shown some idealized simulations using FV3 where grid imprinting appears in many simulations.

Generally speaking, grid imprinting is the presence of artificial behaviors on 
the numerical solution that is associated with the grid employed.
It is important to stress out that other quasi-uniform grids may also suffer from grid imprinting.
For instance, a popular mimetic method, known as TRiSK, was proposed in the literature by
\citet{thuburn:2009} and \citet{ringler:2010} using finite difference and finite volume schemes.
This scheme is designed for general orthogonal grids, such as the Voronoi and icosahedral grids,
and ensures mass and total energy conservation.
This method has been employed in the dynamical core of the Model for Prediction Across Scales (MPAS)
from National Center for Atmospheric Research (NCAR) \citep{skamarock:2012}, 
which intended to work on general Voronoi grids, including locally refined Voronoi grids.
However, the TRiSK scheme is a low-order scheme and also suffers from grid imprinting, \textit{i.e.},
geometric properties of the grid, such as cell alignment, interfere with the method accuracy
\citep{weller:2012, peixoto:2013, peixoto:2016}.

Despite being chosen as the new US global weather prediction model,
there is a lack of numerical studies of the FV3 discretizations in the literature,
especially regarding the grid imprinting problem and its mimetic properties.
Numerical results for the advection equation on the cubed-sphere using the
FV3 dynamical core was presented in \citet{putman:2007} and some shallow-water 
simulations were presented in \citet{harris:2013}, considering cubed-spheres
with local refinement through grid nesting.
From the work \citet{harris:2013} we can notice that the FV3 dynamical
lack convergence on the maximum norm for the shallow-water model considering
the classical balanced geostrophic flow test case from \citet{will:1992}.
The authors attribute these errors to the abrupt change in the grid resolution
near the nested grid, but no quantitative results are shown considering the quasi-uniform grid.
Many other papers available in the literature use the complete FV3 model in three-dimensional
frameworks which make it harder to perform a numerical analysis study due not only to its computational cost
but also due to the complexity of three-dimensional atmospheric models.
There are no detailed works published in intermediate two-dimensional frameworks, using, for 
instance, the shallow-water equations on the sphere.
Even though the advection equation on the sphere plays a key role in the dynamical core development,
since it models the transport of scalar fields on the sphere, important features captured by the shallow-water
equations on the sphere, such as the Coriolis effect, inertia-gravity waves, geostrophic adjustment, Rossby waves,
among others, are not captured by a simple advection model.
Hence, shallow-water equations provide an excellent benchmark to assess dynamical cores in general,
since it is only two-dimensional but is a complex enough geophysical model for atmosphere dynamics.

\section{Outline and contributions}
This thesis is outlined as follows.
\begin{itemize}
\item Chapter \ref{chp-1d-fv} is dedicated to review the Piecewise Parabolic Method (PPM)
for the one-dimensional advection equation. 
In this Chapter, we demonstrate how the temporal component of PPM can be expressed as a departure point calculation.
Subsequently, we enhance the departure point calculation from first-order (which is utilized in FV3) to second-order.
This enhancement results in a significant improvement, particularly in non-constant wind simulations.
Its benefit is also observed when using the monotonic version of PPM used in FV3 and proposed by \citet{lin:2004}.
The additional cost is only due to linear interpolation, which has little impact on the overall performance.
\item Chapter \ref{chp-2d-fv} reviews the dimension splitting method, which allow us to use 
one-dimensional methods, such as the PPM, to solve the two-dimensional advection equation.
We review the current 2D advection scheme of FV3 on the plane proposed by \citet{putman:2007}.
The main feature of this scheme is that it preserves a constant scalar field when the wind is divergence free.
We show through some numerical simulations that this scheme is second-order order accurate only for divergence free winds.
When the wind is not divergence free, we show that this scheme is only first-order accurate.
On the other hand, we propose a small modification of the \citet{putman:2007} scheme using the
second-order departure point computation presented in Chapter \ref{chp-1d-fv}, that allow us to achieve second-order
and smaller errors for divergent and divergence free winds,
despite this scheme does not preserve the a constant scalar for a divergence free, but having a second-order error in this case as well.
Further, when the monotonic scheme is used in the 1D solver,
this scheme has also smaller errors when compared to the  \citet{putman:2007} scheme.
\item Chapter \ref{chp-cs-grids} introduces the cubed-sphere grid used in FV3,
namely the equi-edge \citet{chen:2021} and equiangular grids \citet{ronchi:1996},
and we investigate its geometrical properties. 
Then, we present all the tools needed to extend the advection schemes on the plane from Chapter \ref{chp-2d-fv} to the sphere.
We review the contravariant/covariant wind formulation induced by the cubed-sphere mappings.
Also, we show how we can compute stencils near to the cube-edges through the duo-grid technique to generate the ghost cells that allow us to 
use 1D Lagrange interpolation to fill the ghost cells.

\item Chapter \ref{chp-cs-fv} extends the ideas of Chapter \ref{chp-2d-fv} to the cubed-sphere grid using the tools from Chapter \ref{chp-cs-grids}.
The dimension-splitting method on each cubed-sphere panel works as in the plane, with the addition
of metric terms, due to non-orthogonality of the grid, and interpolation between panels
to obtain ghost cells values needed for stencil computations.
We show that the scheme from \citet{putman:2007} uses a less accurate formulation of the metric term to preserve the constant scalar
for a divergence-free wind, while our new scheme may use a more accurate metric term formulation, since it does not have this preservation constraint.
The results are essentially the same from Chapter \ref{chp-2d-fv},
showing that our scheme is successfully extended from the plane to the cubed-sphere and is more accurate.
We also show that our new scheme has smaller error at the corner compared to the scheme from \citet{putman:2007}.
\end{itemize}
We give some future work perspectives in Chapter \ref{chp-conclusions}.