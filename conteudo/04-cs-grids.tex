\chapter{Cubed-sphere grids}
\label{chp-cs-grids}
So far, we have described the dimension-splitting technique in Chapter \ref{chp-2d-fv} for solving the advection equation on the plane.
Our current goal is to apply these schemes to solve the advection equation on the sphere.
Consequently, we need to introduce a grid over the sphere.
In order to facilitate the extension of dimension-splitting techniques onto the sphere, we require a logical Cartesian coordinate system, at least locally.

We point out that dimension-splitting schemes could be formulated in unstruced grids (see for instance \citet{herzfeld:2023}).
A good reason to use a locally Cartesian grid is to avoid problems, such as the lack of convergence of the divergence operator among others,
that may arise in some grid cells within those grids \citep{peixoto:2013, peixoto:2016, weller:2012}.
Also, a logical Cartesian coordinate system eases the process of higher-order interpolation,
which can be more complicated on a spherical unstructured grid, requiring tangent plane approximations \citep{peixoto:2014,skamarock:2011}.

The scheme proposed by \citet{lin:1996} was originally implemented on latitude-longitude grids,
and the FV dynamical core was elucidated in \citet{lin:2004}.
The latitude-longitude grids exhibit convergence of meridians at the poles, 
necessitating the utilization of the Semi-Lagrangian formulation of PPM for larger CFL numbers,
as discussed in Section \ref{chp-adv1d-sec-flux}, 
to overcome the CFL restriction imposed by the poles.
However, this approach needs the processes in a parallel domain decomposition of the latitude-longitude grid to utilize more
data at the poles, resulting in non-parallel efficiency.
Therefore, \citet{putman:2007} proposed considering the cubed-sphere (CS, hereafter) instead.
The CS grid is more uniform, thus not exhibiting a strong CFL condition anywhere.
This eliminates the need for the Semi-Lagrangian formulation of PPM, which is better to parallel efficiency, and led to the development of the FV3 core.


The CS grid was originally proposed by \citet{sadourny:1972} and was 
reinvestigated by \citet{ronchi:1996} and \citet{rancic:1996}. 
As is usual for Planotic grids, we start with a Platonic solid, in this case, a cube, 
which is circumscribed in a sphere. We then project its faces onto the sphere.
The original CS, called the equidistant CS, was proposed by 
\citet{sadourny:1972} but resulted in a non-uniform grid. 
To address this issue, a solution was proposed by introducing angular coordinates, 
leading to a quasi-uniform grid known as the equiangular CS.
The cubed sphere consists of six panels, each one having a local Cartesian coordinate 
system. 
As we pointed out before, this makes it easier to extend methods from the plane to the sphere. 
In fact, \citet{putman:2007} extends the dimension splitting technique from 
\citet{lin:1996}, as presented in Chapter \ref{chp-2d-fv}, to the CS.

There are essentially two major challenges when working with the CS grid:
\begin{enumerate}
\item
The non-orthogonal grid system: This challenge is primarily related to the appearance of
metric terms in the equations. It adds computational cost and often requires conversions
between contravariant and covariant components of a velocity field.
\item
The discontinuity of the coordinate system at the cube edges: This is perhaps the most 
problematic challenge. Computing stencils along the cube edges becomes challenging due to 
the discontinuous nature of the coordinate system.
\end{enumerate}
One possible approach to compute stencils at the edges is to extend the local coordinate 
of each panel to its neighboring panels, adding ghost cells in the halo region. In the 
case of the equiangular CS, ghost cell values lie on the same geodesics 
containing the data from the neighboring panels. This allows for the use of one-dimensional
high-order Lagrange interpolation to compute the stencils at the edges. 
This approach has been extensively used in the literature \citep{croisille:2013, 
katta:2015, katta:2015b, chen:2021} and was initially introduced by \citet{ronchi:1996}.
This approach is referred to as \textbf{duo-grid}, as named by \citet{chen:2021}.
Alternatively, \citet{putman:2007} uses extrapolation for the PPM reconstruction values near the cube edges.
Another approach that avoids the need for interpolation or extrapolation near the edges is 
the conformal CS developed by \citet{rancic:1996}. While this grid leads to an 
orthogonal and continuous coordinate system near the edges, it generates grid singularities
near the cube corners, similar to the pole problem. 
An improved and more uniform conformal grid, called the Uniform Jacobian cubed sphere, was 
later proposed by \citet{rancic:2017}.
Each approach is likely to generate grid imprinting, and one of the goals of this work is 
to investigate the amount of grid imprinting produced by each method.

This chapter aims to review and investigated the geometrical properties
of the CS. Besides that, we also aim to investigate the process
of interpolating/extrapolating near the cube edges.
We start with a basic review of the CS mappings in Section \ref{cs-mappings},
while Section \ref{cs-halodata} investigates how we can apply 1D Lagrange interpolation using the adjacent panels
data to obtain values of a scalar/vector field on ghost cells, as well different edge
treatments when computing stencils near to the cube edges.
Final thoughts are presented in Section \ref{cs-conc}.


\section{Cubed-sphere mappings}
\label{cs-mappings}

\section{Edges treatment}
\label{cs-halodata}

\section{Concluding remarks}
\label{cs-conc}
